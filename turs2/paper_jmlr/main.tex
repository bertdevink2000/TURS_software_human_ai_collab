\documentclass[twoside,11pt]{article}

\usepackage{algorithm}
\usepackage{algorithmic}
\usepackage{xcolor}

\renewcommand{\algorithmiccomment}[1]{\textcolor{blue}{\textit{\# #1}}} % Set comment text to blue and italic
\renewcommand{\algorithmicrequire}{\textbf{INPUT:}}
\renewcommand{\algorithmicensure}{\textbf{OUTPUT:}}

\usepackage{blindtext}
%\usepackage[linesnumbered,ruled,vlined]{algorithm2e}
%\SetKwInput{KwInput}{Input}                % Set the Input
%\SetKwInput{KwOutput}{Output}   
\usepackage{todonotes}
\usepackage{graphicx}
\usepackage{amsmath}
\usepackage{mathrsfs}

\usepackage{thmtools,thm-restate}


           % set the Output
\newcommand{\htheta}{\hat{\theta}}
\newcommand{\ruleset}{M}
\newcommand{\assign}{\leftarrow}

\renewcommand\vec{\boldsymbol}

% Any additional packages needed should be included after jmlr2e.
% Note that jmlr2e.sty includes epsfig, amssymb, natbib and graphicx,
% and defines many common macros, such as 'proof' and 'example'.
%
% It also sets the bibliographystyle to plainnat; for more information on
% natbib citation styles, see the natbib documentation, a copy of which
% is archived at http://www.jmlr.org/format/natbib.pdf

% Available options for package jmlr2e are:
%
%   - abbrvbib : use abbrvnat for the bibliography style
%   - nohyperref : do not load the hyperref package
%   - preprint : remove JMLR specific information from the template,
%         useful for example for posting to preprint servers.
%
% Example of using the package with custom options:
%
% \usepackage[abbrvbib, preprint]{jmlr2e}

\usepackage{jmlr2e}

% Definitions of handy macros can go here

\newcommand{\dataset}{{\cal D}}
\newcommand{\fracpartial}[2]{\frac{\partial #1}{\partial  #2}}

% Heading arguments are {volume}{year}{pages}{date submitted}{date published}{paper id}{author-full-names}

\usepackage{lastpage}
\jmlrheading{23}{2022}{1-\pageref{LastPage}}{1/21; Revised 5/22}{9/22}{21-0000}{Yang L. and van Leeuwen M.}

% Short headings should be running head and authors last names

\ShortHeadings{Truly Unordered Decision Rules}{Yang L. and van Leeuwen M.}
\firstpageno{1}

\begin{document}

%\title{Truly Unordered Decision Rules for Interpretable Machine Learning with a Probabilistic Approach}
%\title{Truly Unordered Decision Rules}
\title{Probabilistic Decision Rules for Multi-class Classification with Truly Unordered Rules}

\author{\name Lincen Yang \email l.yang@liacs.leidenuniv.nl \\
       \addr LIACS, 
       Leiden University\\
       Leiden, The Netherlands
       \AND
       \name Matthijs van Leeuwen \email m.van.leeuwen@liacs.leidenuniv.nl \\
       \addr LIACS, 
       Leiden University\\
       Leiden, The Netherlands}

\editor{My editor}

\maketitle

\begin{abstract}%   <- trailing '%' for backward compatibility of .sty file
Rule set learning has long been studied and has recently been frequently revisited due to the need for interpretable models. However, existing methods have several shortcomings. First, existing methods impose orders among rules, either explicitly or implicitly, which makes the model less comprehensible by domain experts. Further, due to the difficulty of handling probabilistic conflicts caused by overlaps (instances covered by multiple rules), existing methods often do not consider \emph{probabilistic} rules, although probabilistic outputs of rules carry the uncertainty of predictions and hence are more informative in describing subgroups in datasets. Besides, the divide-and-conquer strategy and learning from pre-mined rules are the two most widely used approaches, which can only be used for binary classification or for multi-class classification via the ``one-versus-rest" approach, while directly learning decision rules for multi-class target is understudied. 

We propose TURS, for Truly Unordered Rule Sets, to address these shortcomings. We first formalize the problem of learning truly unordered rule sets. To resolve conflicts caused by overlapping rules, we propose a novel approach that exploits the probabilistic properties of our rule sets, with the intuition by only allowing rules to overlap if they have similar probabilistic output. We next develop a heuristic algorithm that learns rule sets by carefully and extensively extending the common used heuristics. We benchmark against a wide range of rule-based methods and demonstrate that TURS learns rule sets that have both better interpretability and better predictive performance. 
%Finally, we also illustrate that rule sets induced by TURS are empirically truly unordered. 


%An important innovation is that we use an auxiliary beam with a surrogate score to take the global potential of the rule set into account when learning a local rule.



\end{abstract}

\begin{keywords}
  Decision rule, interpretable machine learning, unordered rule set
\end{keywords}

\section{Introduction} \label{sec:intro}
When using predictive models in sensitive real-world scenarios, such as in health care, analysts seek for intelligible and reliable explanations for predictions. Classification rules have considerable advantages here, as they are directly readable by humans. While rules all seem alike, however, some are more interpretable than others. The reason lies in the subtle differences of how rules form a model. Specifically, rules can form an unordered \emph{rule set}, or an explicitly ordered \emph{rule list}; further, they can be categorized as probabilistic or non-probabilistic. 

%In practice, probabilistic and unordered rule sets should be preferred, for the following reasons. First, when a human is responsible to make the final decision, predictions \emph{with} probability estimates provide more information w.r.t.\ the expected ``utility". Next, rule lists are more difficult to interpret than rule sets, as the interpretation of any rule depends on all preceding rules. 
In practice, probabilistic rules should be preferred because they provide information about the uncertainty of the predicted outcomes, and thus are useful when a human is responsible to make the final decision, as the expected ``utility" can be calculated. 
%1) when a human is responsible to make the final decision, predictions \emph{with} probability estimates provide more information w.r.t.\ the expected ``utility", and 2) it provides more information about how a certain predicted outcome is produced, which increases the algorithm transparency.
 Meanwhile, unordered rule sets should also be preferred, as they have better properties regarding interpretability than ordered rule lists. 
 While no agreement has been reached on the precise definition of interpretability of machine learning models 
 \citep{murdoch2019interpretable,molnar2020interpretable}, we specifically treat interpretability with domain experts in mind. From this perspective, a model's interpretability intuitively depends on two aspects: the degree of difficulty for a human to comprehend the model itself, and to understand a single prediction. Unordered probabilistic rule sets are favorable with respect to both aspects, for the following reasons. First, comprehending ordered rule lists requires comprehending not only each individual rule, but also the relationship among the rules, while comprehending unordered rule sets requires only the former. Second, the explanation for a single prediction of an ordered rule list must contain the rule that the instance satisfies, together with all of its preceding rules, which becomes incomprehensible when the number of preceding rules is large. 

Further, crucially, existing methods for rule set learning claim to learn unordered rule sets, but most of them are not truly unordered. The problem is caused by \emph{overlap}, i.e., a single instance satisfying multiple rules. Ad-hoc schemes are widely used to resolve prediction conflicts caused by overlaps, typically by ranking the involved rules with certain criteria and always selecting the highest ranked rule \citep{zhang2020diverseRuleSets,lakkaraju2016interpretable} (e.g., the most accurate one). This, however, imposes implicit orders among rules, making them entangled instead of truly unordered. 

This can badly harm interpretability: to explain a single prediction for an instance, it is now insufficient to only provide the rules the instance satisfies, because other higher-ranked rules that the instance does \emph{not} satisfy are also part of the explanation. For instance, imagine a patient is predicted to have \emph{Flu} because they have \emph{Fever}. If the model also contains the higher-ranked rule \emph{``Blood in stool $\rightarrow$ Dysentery"}, the explanation should include the fact that \emph{``Blood in stool"} is not true, because otherwise the prediction would change to \emph{Dysentery}. If the model contains many rules, it becomes impractical to have to go over all higher-ranked rules for each prediction. 

Learning truly unordered probabilistic rule sets is a very challenging problem though. Classical rule set learning methods usually adopt a separate-and-conquer strategy, often sequential covering: they iteratively find the next rule and remove instances satisfying this rule. This includes 1) binary classifiers that learn rules only for the ``positive" class \citep{furnkranz2012foundations}, and 2) its extension to multi-class targets by the one-versus-rest paradigm, i.e., learning rules for each class one by one \citep{cohen1995ripper,clark1991cn2Improve}. Importantly, by iteratively removing instances the \emph{probabilistic predictive conflicts} caused by overlaps, i.e., rules having different probability estimates for the target, are ignored. Recently proposed rule learning methods go beyond separate-and-conquer by leveraging discrete optimization techniques \citep{zhang2020diverseRuleSets,wang2017bayesian,yang2021learning,lakkaraju2016interpretable,dash2018boolean}, but this comes at the cost of requiring a binary feature matrix as input. Moreover, these methods are neither probabilistic nor truly unordered, as they still use ad-hoc schemes to resolve predictive conflicts caused by overlaps. 

\emph{Approach and contributions.}
To tackle these challenges and learn truly unordered probabilistic rules, we propose a novel approach that we only allow overlaps that are formed by rules with ``similar" probabilistic output. In this case, when an individual instance is covered by multiple rules, the conflict caused by different rules is minimized. 

In practice, to characterize the similarity among rules' probabilistic outputs, we need to trade off between three quantities: the differences between the probability estimates, the size of the overlap, and the model complexity. Instead of explicitly defining hyper-parameters to control the trade off, we take a principled model selection approach, for which we formally define rule sets as probabilistic models. Specifically, we propose an effective and novel approach to formalize the conditional class probability estimates: informally, we take the union for the intersection. 

We further design a model selection criterion based on the minimum description length (MDL) principle \citep{grunwald2019minimum}, which does not require a regularization parameter to be tuned. 
We resort to heuristics for optimization as the search space combined with the model selection criterion do not allow efficient search. Yet, we carefully and extensively extend the common heuristic approach for learning decision rules from data, in the following aspects. First, we consider a ``learning rate" heuristic, i.e., the decrease of our optimization score (to be minimized) \emph{per extra covered instance} as the quality measure for \emph{potentially incomplete rule sets}. Second, we take a novel beam search approach, such that 1) the degree of ``patience" is considered by using a diverse beam search approach, and 2) an auxiliary beam together with a surrogate score is proposed, in order to resolve the issue that, as a unique challenge coming along with our model formalization that allows overlap, rules that have been added to the rule set may become obstacles for new rules. 
Third, a local constraint is used for ``looking-ahead" on the potential for the instances being left out when rules are being refined (i.e., when more literals are added to the rules). 

Finally, we benchmark our algorithm, named TURS, for Truly Unordered Rule Sets with extensive empirical comparisons against a wide range of rule-based methods. We show that TURS has superior performance in the following aspects: 1) TURS has very competitive predictive performance measured by ROC-AUC; 2) TURS can empirically learn truly unordered rules: the probabilistic conflicts caused by overlaps are minimized, in the sense that the influence is little even if we predict for instances covered by multiple rules by randomly picking a rule from these rules; 3) TURS learns a set of decision rules with probability estimates for the target variable that can generalize well to unseen data; and 4) TURS produces simpler models in comparison to competitor algorithms. 

%We adopt a probabilistic model selection approach for rule set learning, for which we design a criterion based on the minimum description length (MDL) principle \citep{grunwald2019minimum}. 

%Second, we propose a novel surrogate score based on decision trees that we use to evaluate the potential of incomplete rule sets. Third, we are the first to design a rule learning algorithm that deals with probabilistic conflicts caused by overlaps already during the rule learning process. We point out that rules that have been added to the rule set may become obstacles for new rules, and hence carefully design a two-phase heuristic algorithm, for which we adopt diverse beam search \citep{matthijs2012diverse}. 

%Last, we benchmark our method, named \textsc{Turs}, for Truly Unordered Rule Sets, against a wide range of methods. We show that the rule sets learned by \textsc{Turs}, apart from being probabilistic and truly unordered, have better predictive performance than existing rule list and rule set methods. 



\section{Related Work} \label{sec:related}

\noindent
\emph{Rule lists.}
%Rules in a rule list are connected by \textsc{if-then-else} statements. Existing methods include ordered-CN2~\citep{clark1989cn2}, CBA~\citep{liu1998CBA}, and PART~\citep{frank1998generating}, as well as the recently proposed CLASSY~\citep{proencca2020interpretable} and Bayesian rule list~\citep{yang2017scalable-bayesian-rule-list}. Rule lists are more difficult to interpret than rule sets because of their explicit orders. 
Rules in a rule list are connected by \textsc{if-then-else} statements. Existing methods include CBA~\citep{liu1998CBA}, ordered CN2~\citep{clark1989cn2}, PART~\citep{frank1998generating}, and the more recently proposed CLASSY~\citep{proencca2020interpretable} and Bayesian rule list~\citep{yang2017scalable-bayesian-rule-list}. We argue that rule lists are more difficult to interpret than rule sets because of their explicit orders. 

\smallskip \noindent
\emph{One-versus-rest learning.}
This category focuses on only learning rules for a single class label, i.e., the ``positive" class, which is already sufficient for binary classification \citep{wang2017bayesian,dash2018boolean,yang2021learning}. For multi-class classification, two approaches exist. The first, taken by RIPPER~\citep{cohen1995ripper} and C4.5~\citep{quinlan2014c4}, is to learn each class in a certain order. After all rules for a single class have been learned, all covered instances are removed (or those with this class label). The resulting model is essentially an ordered list of rule sets, and hence is more difficult to interpret than rule set.

The second approach does not impose an order among the classes; instead, it learns a set of rules for each class against all other classes. The most well-known are unordered-CN2 and FURIA~\citep{clark1991cn2Improve,huhn2009furia}. 
%However, they do not learn truly unordered rule sets, as discussed in Section~\ref{sec:intro}. 
FURIA avoids dealing with conflicts of overlaps by essentially using all (fuzzy) rules for predicting unseen instances; i.e., the rules' outputs are weighted by the so-called ``certainty factor". As a result, it cannot provide a single rule to explain its prediction. Unordered-CN2, on the other hand, handles overlaps by ``combining" all overlapping rules into a ``hypothetical" rule, which sums up all instances in all overlapping rules and hence ignoring probabilistic conflicts for constructing rules. In Section~\ref{sec:exp}, we show that our method learns smaller rule sets with better predictive performance than unordered-CN2.

\smallskip \noindent
\emph{Multi-class rule sets.}
Very few methods exist for directly learning rules for multi-class targets, which is algorithmically more challenging than the one-versus-rest paradigm, as the separate-and-conquer strategy is not applicable. 
To the best of our knowledge, the only existing methods are IDS~\citep{lakkaraju2016interpretable} and DRS~\citep{zhang2020diverseRuleSets}. Both are neither probabilistic nor truly unordered. To handle conflicts of overlaps, IDS follows the rule with the highest F1-score, and DRS uses the most accurate rule. 
%Two existing methods IDS~\citep{lakkaraju2016interpretable} and DRS~\citep{zhang2020diverseRuleSets} are neither probabilistic, nor truly unordered: to handle predictive conflicts of overlaps, IDS follows the rule with the highest F1-score, and DRS uses the most accurate rule. 
%Decision tree methods like CART~\citep{breiman1984classification} might also fall in this category as well, but rules based on decision trees are forced to share many ``attributes", and hence become longer than necessary. 

Last, different but related approaches include the following. Firstly,  decision tree based methods such as CART~\citep{breiman1984classification}, which produce rules that are forced to share many ``attributes" and hence are longer than necessary, as we will empirically demonstrate in Section~\ref{sec:exp}. Secondly, the Bayesian rule mining~\citep{gay2012bayesian} method, which adopts naive bayes  with the mined rules for prediction, and hence does not produce a rule set model in the end. Third, the `lazy learning' approach for rule-based models can also avoid the conflicts of overlaps~\citep{veloso2006lazy}, but no global rule set model describing the whole dataset is constructed in this case.

%\smallskip \noindent
%\emph{Bayesian and MDL rule learning.} Last, we mention previous rule learning methods that also adopts a model selection approach with Bayesian or MDL criterion~\citep{wang2017bayesian, proencca2020interpretable, }.

% \subsection{Rule Sets} \label{subsec:related_rulesets}
% \subsubsection{Rule Sets from Decision Trees}
% Any decision tree method can in principle be used as rule induction method, as each path of a decision tree is a decision rule. As no two paths can overlap, each induced rule can be interpreted individually. Numerous decision tree methods exist \citep{kotsiantis2013decision}, where representatives include the classic ones such as CART~\citep{breiman1984classification} and C4.5~\citep{quinlan2014c4}. Nijssen and Fromont proposed a bottom-up way to form decision trees and showed the close connection between decision trees and itemset mining \citep{nijssen2010optimal}.

% The biggest drawback of rules induced by trees is that the tree structure forces rules (i.e., the paths) to share many features, which makes rules longer than necessary, even if the complexity of trees is controlled, e.g., via pruning. Quinlan~\citep{quinlan1987simplifying} proposed a method for simplifying rules induced by trees; however, this causes overlap and an ad-hoc strategy is used to resolve conflicts. 

% \subsubsection{Rule Sets for Binary Targets}
% Many rule set methods for binary classification tasks exist \citep{wang2017bayesian,dash2018boolean}. In particular, if rules are constrained to only cover one class (e.g., the positive class) and always leave the other class to the ``else-rule'', overlapping rules will not cause any conflicts. These methods are clearly not applicable to the more general case of multi-class classification, unless separate ensemble methods are leveraged. Model ensembles are less interpretable than a single model though. For instance, Wang et al.~\citep{wang2017bayesian} suggest to use error-correcting output codes \citep{dietterich1994errorCorrecting}, which requires a distance measure to assign a new data point to the closest code words of the code---this harms explainability. 

% \subsubsection{Rule Sets that allow Overlap}
% Contrary to our approach, which treats the whole rule set as a single probabilistic model that can model any (seen or unseen) data point in a principled way, previous rule set learning methods that allow overlapping rules all require a separate scheme for conflict resolution. unorderer-CN2 \citep{clark1991cn2Improve}, for example, iteratively treats each class label as the ``positive'' class, and uses a sequential covering strategy that removes any covered `positive' data points. In case of conflicts caused by overlap, it uses the union of all involved rules to make the prediction. 
% %Note that although we also use a sequential covering strategy for learning, we do not need a separate strategy for conflict resolution as this is part of our (probabilistic) model (and thus of model selection).

% Bostr{\"o}m~\citep{bostrom2004pruning} proposed to predict the class label of a data point covered by multiple rules using a na\"ive Bayes approach, which calculates the probability of the label given the rules by first calculating the probability of each rule given the label. Following this line, methods using intersecting rules \citep{lindgren2002classification} were proposed, which basically treats each overlap as an intersection (conjunction) of rules; however, this approach potentially creates a huge number of implicit, derived rules and may cause serious overfitting. 

% FURIA~\citep{huhn2009furia} is a fuzzy rule set method, where the conflicts of overlaps are handled by predicting the class label with the highest frequency; hence, FURIA has no probabilistic interpretation. Further, IDS~\citep{lakkaraju2016interpretable} optimizes a linear combination of several scores, which aim for characterizing different aspects of the quality of a rule set, and hence involves extensive hyper-parameter tuning for the ``weights" of this linear combination. To resolve possible conflicts caused by overlap, IDS uses the rule with the highest F1-score. 

% Finally, the very recently proposed DRS~\citep{zhang2020diverseRuleSets} formalizes the problem of rule set learning as a regularized optimization problem that explicitly penalizes large overlaps of rules, and develops a heuristic to choose the regularization parameter to control the degree of penalty without cross-validation. In case of conflicts caused by overlapping rules, DRS suggests to use the most accurate rule. 
%Algorithmically, DRS first reduces the search space by sampling a from all candidate rules, and then adopts a heuristic algorithm to search for the final rule set, which not only learns worse rule sets than our algorithm but also is empirically more time consuming, as shown in Section~\ref{sec:experiments}.

% \paragraph{Other related methods}

% Another related but different approach are rule ensembles, i.e., methods that treat rules as base learners and propose ensemble methods (usually with boosting) for prediction \citep{cohen1999slipper, friedman2008predictive, boley2021better}. However, the interpretability of rule ensembles is not comparable to that of rule sets \citep{dash2018boolean}, as each rule merely becomes a base learner and ``one step'' in a stage-wise optimization process~\citep{friedman2001greedy}. 
%\subsection{Interpretability}
%\subsection{Optimization of Rule Learning and Hardness Results}



\section{Rule Sets as Probabilistic Models}\label{section:rulesetModel}

We first formalize individual rules as \emph{local} probabilistic models, and then define rule sets as \emph{global} probabilistic models. The key challenge lies in how to define $P(Y=y|X=x)$ for an instance $(x,y)$ that is covered by multiple rules. 
% We start with introducing notation and defining classification rules from a probabilistic perspective, and discuss the two competing factors that determine rule quality: \emph{approximation accuracy} and \emph{coverage}. We next describe how we can treat any rule set as a probabilistic model for a dataset, in particular when rules are potentially overlapping. This includes parameter estimation and predicting a probability distribution over class labels for any data point. 

\subsection{Probabilistic Rules}
Denote the input random variables by $X = (X_1, \ldots, X_d)$, where each $X_i$ is a one-dimensional random variable representing one dimension of $X$, and denote the categorical target variable by $Y\in\mathscr{Y}$. Further, denote the dataset from which the rule set can be induced as $D = \{(x_i, y_i)\}_{i \in [n]}$, or $(x^n, y^n)$ for short. Each $(x_i,y_i)$ is an instance. Then, a probabilistic rule $S$ is written as
\begin{equation}
(X_1 \in R_1 \land X_2 \in R_2 \land \ldots) \rightarrow P_S(Y),
\end{equation}
where each $X_i \in R_i$ is called a \emph{literal} of the \emph{condition} of the rule. Specifically, each $R_i$ is an interval (for a quantitative variable) or a set of categorical levels (for a categorical variable). 

A probabilistic rule of this form describes a subset $S$ of the full sample space of $X$, such that for any $x \in S$, the conditional distribution $P(Y | X=x)$ is approximated by the probability distribution of $Y$ conditioned on the event $\{X \in S\}$, denoted as $P(Y | X \in S)$. Since in classification $Y$ is a discrete variable, we can parametrize $P(Y|X\in S)$ by a parameter vector $\vec{\beta}$, in which the $j$th element $\beta_j$ represents $P(Y=j|X\in S)$, for all $j \in \mathscr{Y}$. We therefore denote $P(Y | X \in S)$ as $P_{S, \vec{\beta}}(Y)$, or $P_S(Y)$ for short. To estimate $\vec{\beta}$ from data, we adopt the maximum likelihood estimator, denoted as $P_{S, \hat{\vec{\beta}}}(Y)$, or $\hat{P}_S(Y)$ for short.
% For multi-class classification, $Y$ always takes value in a finite set of integers. Hence $P(Y|X\in S)$ always follows a categorical distribution, which can be parameterized as $P_{S, \beta}(Y)$, where $\beta$ is simplex that represents the probability of $Y$ equal to each class. 

Further, if an instance $(x,y)$ satisfies the condition of rule $S$, we say that $(x,y)$ is \emph{covered} by $S$. Reversely, the \emph{cover} of $S$ denotes the instances it covers. When clear from the context, we use $S$ to both represent the rule itself and/or its cover, and define the number of covered instances $|S|$ as its \emph{coverage}. 

% In practice, $P_S(Y)$ can be estimated from data by the maximum likelihood estimator, with all instances covered by $S$. Formally, if $Y$ takes values in $\{1, 2, \ldots, K\}$ (in short $[K]$), 
% \begin{equation}
% \hat{P}_S(Y = k) = \frac{|\{(x,y): x \in S, y = k\}|}{|S|}, k \in [K]   
% \end{equation}
% As we only consider classification tasks, i.e., $Y$ is categorical, we assume that for any $X \in S$, $P(Y|X)$ can be approximated by a single multinomial distribution. 

% Thus, a probabilistic rule is a local multinomial model that covers a subset $S$ of the full sample space, with the multinomial parameters to be estimated from data $(x^n, y^n)$. Formally, we denote the multinomial model associated with rule $S$ as a family of multinomial distributions $P_{X\in S, \theta_S}(Y)$ indexed by $\theta_S$, or in short $P_S(Y)$. Note that 1) if a data point $(x,y)$ satisfies $x \in S$, we say that $(x,y)$ or $x$ is \emph{covered} by rule $S$; and 2) we use $S$ to both represent a rule and its corresponding subset of the sample space. 

% Intuitively, it is clear that for a rule to be of good `quality', the conditional probability distribution $P(Y|X=x)$ should not vary drastically for the different data points it covers, i.e., it should be similar for all $x \in S$. On the other hand, however, a probabilistic rule needs to cover sufficiently many data points to be able to reliably estimate $P_S(Y)$, to generalize well and thus avoid overfitting, and to be meaningful for knowledge discovery. That is, formally, the quality of a probabilistic rule depends on two factors: 1) \emph{approximation accuracy}, i.e., how accurately the associated $P_S(Y)$ of a rule approximates $P(Y|X=x)$ when the rule covers $x$; and 2) \emph{coverage}, i.e., how many data points the rule covers. These two factors typically compete in practice and hamper us from simply ranking all probabilistic rules and defining an objectively optimal rule. Instead, the task of learning high-quality rules is all about finding the ``best" trade-off between approximation accuracy and coverage, which is achieved by learning a set of rules. 

\subsection{Truly Unordered Rule Sets as Probabilistic Models} \label{subsec:probRuleSet}
% We now formulate rule sets as probabilistic models: we show how given a rule set  $\ruleset$ and a dataset $(x^n, y^n)$, we can calculate the probabilities of class labels $y^n$ given features $x^n$, denoted $P(y^n|x^n)$. 

While a rule set is simply a set of rules, the challenge lies in how to define rule sets as probabilistic models while keeping the rules truly unordered. That is, how do we define $P(Y|X=x)$ given a rule set $\ruleset$ together with associated parameters. 

That is, how do we define $P(Y|X=x)$ given a rule set $\ruleset$? We first explain how to do this for a single instance of the training data, using a simplified setting where at most two rules cover the instance. We then discuss---potentially unseen---test instances and extend to more than two rules covering an instance. Finally, we define a rule set as a probabilistic model.


\medskip \noindent
\textbf{Class probabilities for a single training instance.} 
Given a rule set $\ruleset$ with $K$ individual rules, denoted $\{S_i\}_{i \in [K]}$, any instance $(x,y)$ falls into one of three cases: 1) exactly one rule covers $x$; 2) at least two rules cover $x$; and 3) no rule in $M$ covers $x$.

To simplify the notation, we here consider at most two rules covering an instance, which can be trivially extended to the case when the overlap contains multiple rules.

\smallskip \noindent
\emph{Covered by one rule.} 
% When exactly one rule $S_i$ covers $x$, we use the multinomial model associated with $S_i$, denoted as $P_{S_i, \theta_{S_i}}$, to model the probability of a class label $y$. Here the multinomial parameter $\theta_{S_i}$ is estimated based on the subset of $(x^n, y^n)$ covered by $S_i$, denoted as $(x^{|S_i|}, y^{|S_i|})$, using the maximum likelihood (ML) estimator. Formally the ML estimator is $\theta^*_{S_i} = (h_1/|S_i|, \ldots, h_K/|S_i|)$ \cite{casella2021statistical}, where $|S_i|$ denotes the coverage (i.e., the number of covered data points) of $S_i$, and each $h_k \in (h_1, \ldots, h_K)$ is the number of $y \in y^n$ such that $y$ is equal to the class label $k$, i.e., formally $h_k$ is the cardinality of $\{y \in y^n| y = k\}$. In other words, we use the cover of a rule $S_i$ to approximate $P(Y|X=x)$ by $P_{S_i, \theta^*_{S_i}}$. 
When exactly one rule $S \in \ruleset$ covers $x$, we use $P_S(Y)$ to ``approximate" the conditional probability $P(Y|X=x)$. To estimate $P_S(Y)$ from data, we adopt the maximum likelihood (ML) estimator $\hat{P}_S(Y)$, i.e.,
%denoted $\hat{P}_S(Y)$ and defined as
\begin{equation}
\hat{P}_S(Y = j) = \frac{|\{(x,y): x \in S, y = j\}|}{|S|}, \forall j \in \mathscr{Y}. 
\end{equation}
% Note that even if $S_i$ overlaps with, for instance, a rule $S_j$ (or possibly more rules), we use \emph{all} data points covered by $S_i$ to estimate the multinomial parameter of rule $S_i$---including those data points also covered by $S_j$. That is, we do \emph{not} (implicitly) use the multinomial model associated with $S_i \setminus S_j$, but the multinomial model associated with the full $S_i$. By doing so, we guarantee that it is only the explicit rules in the rule set that model the data, rather than implicitly \emph{derived rules} such as $S_i \setminus S_j$.
Note that we do not exclude instances in $S$ that are also covered by other rules (i.e., in overlaps) for estimating $P_S(Y)$. Hence, the probability estimation for each rule is independent of other rules; as a result, each rule is \emph{self-standing}, which forms the foundation of a truly unordered rule set. 

\smallskip \noindent
\emph{Covered by two rules.}
Next, when $x$ is covered by $S_i$ and $S_j$, we approximate $P(Y|X=x)$ by $P(Y|X \in S_i \cup S_j)$, and we estimate this with its ML estimator $\hat{P}(Y|X \in S_i \cup S_j)$, using all instances in $S_i \cup S_j$. That is, we take the \emph{union} for estimating the \emph{intersection}. 

As counterintuitive as it may seem, this is the key component to formalize the truly unordered rule set model. When $P_{S_i}(Y)$ and $P_{S_j}(Y)$ are very similar, the conditional probability conditioned on the event $\{S_i \cup S_j\}$, denoted as $P(Y|S_i \cup S_j)$, will also be similar to both $P_{S_i}(Y)$ and $P_{S_j}(Y)$. In this case, it does not matter which of these three ($P_{S_i}(Y)$, $P_{S_j}(Y)$, or $P(Y|S_i \cup S_j)$) we use to estimate $P(Y|X=x)$, in the sense that the ``goodness-of-fit" measured by the likelihood of all instances covered by the overlap $S_i \cap S_j$ would be all similar. 

On the other hand, when $P_{S_i}(Y)$ and $P_{S_j}(Y)$ are very different, the goodness-of-fit would be poor when using $P(Y|S_i \cup S_j)$ for estimating $P(Y|X=x)$, where $x \in S_i \cap S_j$. Thus, we leverage this property to penalize ``bad" overlaps by incorporating the probabilistic goodness-of-fit in our model selection criterion discussed in the following Section. 

Intuitively, this simple yet effect approach can automatically balance between the degree of the probability estimates difference (between $P_{S_i}(Y)$ and $P_{S_j}(Y)$) and the size (number of instances) of the overlap $S_i \cap S_j$. It is particularly useful when the estimator of $P(Y|X \in S_i \cap S_j)$, i.e., conditioned on the ``intersection" event $\{X \in S_i \cap S_j\}$, is indistinguishable from $\hat{P}(Y|X \in S_i)$ and $\hat{P}(Y|X \in S_j)$: either because 1) $S_i \cap S_j$ consists of very few instances, so the variance of the estimator for $P(Y|X \in S_i \cap S_j)$ is large, or 2) $P(Y|X \in S_i \cap S_j)$ is just very similar to $P(Y|X \in S_i)$ and $P(Y|X \in S_i)$, which makes it undesirable to create a separate rule for $S_i \cap S_j$. 



%%When an instance is covered by two non-nested, partially overlapping rules, we interpret this as probabilistic \emph{uncertainty}: we cannot tell whether the instance belongs to one rule or the other, and therefore approximate its conditional probability by the \emph{union} of the two rules. That is, in this case we approximate $P(Y|X=x)$ by $P(Y|X \in S_i \cup S_j)$, and we estimate this with its ML estimator $\hat{P}(Y|X \in S_i \cup S_j)$, using all instances in $S_i \cup S_j$. 

%This approach is particularly useful when the estimator of $P(Y|X \in S_i \cap S_j)$, i.e., conditioned on the event $\{X \in S_i \cap S_j\}$, is indistinguishable from $\hat{P}(Y|X \in S_i)$ and $\hat{P}(Y|X \in S_j)$. Intuitively, this can be caused by two reasons: 1) $S_i \cap S_j$ consists of very few instances, so the variance of the estimator for $P(Y|X \in S_i \cap S_j)$ is large; 2) $P(Y|X \in S_i \cap S_j)$ is just very similar to $P(Y|X \in S_i)$ and $P(Y|X \in S_i)$, which makes it undesirable to create a separate rule for $S_i \cap S_j$. Our model selection approach, explained in Section~\ref{sec:model_selection}, will ensure that a rule set with non-nested rules has high goodness-of-fit only if this `uncertainty' is indeed the case. 

%%In this case, instead of using the multinomial model associated with the intersection of $S_i$ and $S_j$ to model $(x,y)$, which would essentially create an implicit new rule, we use the multinomial model associated with the \emph{union} of $S_i$ and $S_j$. 

%% We have $x \in S_i \land x \in S_j$ and create a corresponding multinomial model $P_{S_i \land S_j, \theta_{S_i \land S_j}}$. Then we estimate the multinomial parameter $\theta_{S_i \land S_j} (Y)$ by the maximum likelihood estimator using \emph{all data points within $S_i \cup S_j$}, denoted $\theta_{S_i \land S_j}^*$. Note that $\theta_{S_i \land S_j}^*$ is estimated from data points covered by $S_i \cup S_j$ rather than $S_i \cap S_j$.

% The rationale for this is that we interpret overlap between rules as \emph{uncertainty}: it is unclear whether $x$ `belongs' to $S_i$ or $S_j$. That is, rather than implicitly creating a new \emph{derived rule} by merging $S_i$ and $S_j$, which would be equivalent to $S_i \land S_j$, we interpret this situation as both rules covering $x$, i.e., $x \in S_i$ \textbf{or} $x \in S_j$. With this interpretation, data points in $S_i \cap S_j$ are separately modelled only if we have an explicit rule in the rule set that covers $S_i \cap S_j$.

% Formally, our interpretation corresponds to the situation where the multinomial distribution $\scriptstyle P_{S_i \land S_j, \htheta_{S_i \land S_j}} (Y)$ is not distinguishable from $\scriptstyle P_{S_i, \theta^*_{S_i}} (Y)$ and $\scriptstyle P_{S_j, \theta^*_{S_j}} (Y)$, where $\scriptstyle \htheta_{S_i \land S_j}$, different from $\scriptstyle \theta_{S_i \land S_j}^*$, is estimated from all data points in $S_i \cap S_j$. Intuitively, this can be caused by two reasons: 1) very few data points exist in $S_i \cap S_j$, making the uncertainty of the associated multinomial distribution $\scriptstyle P_{S_i \land S_j, \htheta_{S_i \land S_j}}(Y)$ very large; 2) $\scriptstyle P_{S_i \land S_j, \htheta_{S_i \land S_j}}(Y)$ is very similar to $\scriptstyle P_{S_i, \theta^*_{S_i}}(Y)$ and $\scriptstyle P_{S_j, \theta^*_{S_j}}(Y)$. Under these circumstances, we adopt the  ``safe'' way of interpreting the situation, i.e., we treat $x$ as belonging to $S_i$ \textbf{or} $S_j$, and to model the data $(x,y)$ by the union $S_i \cup S_j$. In this case we say that \emph{$x$ is modeled by the disjunction of rules $S_i$ and $S_j$} and approximate $P(Y=y|X=x)$ by $\scriptstyle P_{S_i \land S_j, \htheta_{S_i \land S_j}} (y)$.

%Although using the intersection of rules is apparently a bad idea, as the number of implicit rules would grow exponentially and cause overfit, one may wonder, what is the rationale of using the union of rules? 

%\smallskip \noindent
%\emph{Covered by two nested rules.}
%When $x$ is covered by both $S_i$ and $S_j$, and $S_i$ is a subset of $S_j$, i.e., $x \in S_i \subseteq S_j$, the rules are nested\footnote{Note that ``nestedness'' is based on the rules' covers rather than on their conditions. For instance, if $S_i$ is $X_1 <= 1$ and $S_j$ is $X_2 <= 1$, $S_i$ and $S_j$ could still be nested.}.
%In this case, we approximate $P(Y|X=x)$ by $P(Y | X \in S_i)$ and interpret $S_i$ as an \emph{exception} of $S_j$. 
%Having such nested rules to model such exceptions is intuitively desirable, as it allows to have general rules covering large parts of the data while being able to model smaller, deviating parts. %Formally, when $S_i \subseteq S_j$ and $P(Y | X \in S_j)$ is significantly different from $P(Y|X \in S_i)$, but is similar to $P(Y | X \in S_j \setminus S_i)$. Then, whether to eliminate the nested overlap by replacing the rule $S_j$ with (possible multiple) rules which (together) cover $S_j \setminus S_i$ becomes a trade off between model complexity and accuracy. Further, from the perspective of statistical hypothesis testing, it is possible that $\hat{P}(Y | X \in S_j)$ and $\hat{P}(Y | X \in S_i)$ are significantly different while $\hat{P}(Y | X \in S_j)$ and $\hat{P}(Y | X \in S_j \setminus S_i)$ are not\footnote{As a illustrative example, assume that the true conditional distribution $P(Y|X=x)$ is the same for all $x \in S_j \setminus S_i$ and follows a Bernoulli distribution, $Y|X=x \sim Bernoulli(0.5)$; also assume that for all $x \in S_i$, $Y|X=x \sim Bernoulli(0.2)$. Thus, $Y|X$ for all $x \in S_j = (S_j \setminus S_i) \cup S_i$ would be a mixture of the previous two Bernoulli distributions. If $S_j \setminus S_i$ contains $1\,000$ data points, and $S_i$ contains $100$ data points, it can be shown by a simulation study that the conditoinal probability distribution  $P(Y | X \in S_i)$ is significantly different from $P(Y | X \in S_j)$, with p-value $= 7.5 \times 10^{-5} \pm 0.001$ (z-test for proportions, averaged on $1 \, 000$ iterations), but $P(Y | x \in S_j \setminus S_i$ is \emph{not} significantly different than $P(Y | X \in S_j)$, with p-value $= 0.304 \pm 0.29$.}.
%%Note that this kind of nested overlap is conceptually possible when we consider the model class of all possible rule sets, and hence the corresponding probabilistic definition is theoretically necessary. Whether rule sets with nested overlaps are always inferior to those without nested overlaps is another problem. 
%In order to preserve the self-standing property of individual rules, for $x \in S_j \setminus S_i$ we still use $P(Y|X \in S_j)$ rather than $P(Y | X \in S_j \setminus S_i)$. Although this might seem counter-intuitive at first glance, using $P(Y | X \in S_j \setminus S_i)$ would implicitly impose an order between $S_j$ and $S_i$, or---equivalently---implicitly change $S_j$ to another rule that only covers instances in $S_j \land \neg S_i$. 
%% Intuitively, nested rules can occur when $S_i$ covers much fewer data points than $S_j$. This is also theoretically and practically possible from the perspective of statistical hypothesis testing\footnote{As a illustrative example, assume that the true conditional distribution $P(Y|X=x)$ is the same for all $x \in S_j \setminus S_i$ and follows a Bernoulli distribution, $Y|X=x \sim Bernoulli(0.5)$; also assume that for all $x \in S_i$, $Y|X=x \sim Bernoulli(0.2)$. Thus, $Y|X$ for all $x \in S_j$ would be a mixture of the previous two Bernoulli distributions. If $S_j \setminus S_i$ contains $1\,000$ data points, and $S_i$ contains $100$ data points, it can be shown by a simulation study that the probability distribution for $Y$ when $x \in S_i$ is significantly different from when $x \in S_j$, with p-value $= 7.5 \times 10^{-5} \pm 0.001$ (z-test for proportions, averaged on $1 \, 000$ iterations), but the probability distribution for $Y$ when $x \in S_j \setminus S_i$ is \emph{not} significantly different than when $x \in S_j$, with p-value $= 0.304 \pm 0.29$}.
 
\smallskip \noindent
\emph{Not covered by any rule.}
When no rule in $M$ covers $x$, we say that $x$ belongs to the so-called ``else rule'' that is part of every rule set and equivalent to $x \notin \bigcup_{i} S_i$. Thus, we approximate $P(Y|X=x)$ by $P(Y | X \notin \bigcup_{i} S_i)$. We denote the else rule by $S_0$ and write $S_0 \in \ruleset$ for the else rule in $\ruleset$. Observe that the else rule is the only rule in every rule set that depends on the other rules and is therefore not self-standing; however, it will also have no overlap with other rules by definition.
%---this is unavoidable if one aims to achieve high accuracy, as the alternative would be to approximate $P(Y|X=x)$ using the entire dataset. 

\medskip \noindent
\textbf{Predicting for a new instance.} \label{subsubsec:new_data}
When an unseen instance $x'$ comes in, we predict $P(Y|X=x')$ depending on which of the aforementioned three cases it satisfies. An important question is whether we always need access to the training data, i.e., whether the probability estimates we obtain from the training data points are sufficient for predicting $P(Y|X=x')$.

Specifically, if $x'$ is covered by $S_i$ and $S_j$, $P(Y|X=x')$ is predicted as $\hat{P}(Y|X \in S_i \cup S_j)$. However, if there are no training data points covered both by $S_i$ and $S_j$, then we would not obtain $\hat{P}(Y|X \in S_i \cup S_j)$ in the training phase. Nevertheless, in this case we have  $|S_i \cup S_j| = |S_i| + |S_j|$, and hence
\begin{equation}
    \hat{P}(Y|X \in S_i \cup S_j) = \frac{|S_i| \hat{P}(Y|X\in S_i) + |S_j| \hat{P}(Y|X\in S_j)}{|S_i| + |S_j|}.
\end{equation}

Thus, if $x'$ is covered by one rule, two rules, or no rule in $M$, the corresponding probability estimates are already obtained during training. Thus, we conclude that access to the training data is not necessary for prediction.

%\medskip \noindent
%\textbf{Extension to overlaps of multiple rules.}
%Whenever an instance $x$ is covered by multiple rules, denoted $J = \{S_i, S_j, S_k, ...\}$, three cases may happen. The first case is all rules in $J$ are nested. Without loss of generality, assume that $S_i \subseteq S_j \subseteq S_k \subseteq ...$; then, following the rationale for case of two nested rules, $P(Y|X=x)$ should be approximated by $P_{S_i}(Y)$. 
%%Next, for all instances in $S_j\setminus S_i$, they must all be in $S_k$ (and all other rules fully covering $S_k$), and hence they also satisfy the case of being covered by multiple nested rules. Following this line, we can recursively simplify the case of multiple nested rules into two nested rules. 
%Therefore, when $x$ is covered by multiple nested rules, only the ``smallest" rule matters and we can act as if $x$ is only covered by that single rule. 
%
%The second case is that all rules in $J$ are non-nested with each other. Following the solution for modeling two non-nested rules, we use $P(Y|X \in \bigcup_{S\in J} S)$. 
%
%The third case is a mix of the previous two cases, i.e., rules in $J$ are partially nested. In this case, we iteratively go over all $S \in J$: if there exists an $S' \in J$ satisfying $S' \subseteq S$ we remove $S$ from $J$, and continue iterating until no nested overlap in $J$ remains. If one single rule is left, we act as if $x$ is covered by that single rule; otherwise, we follow the paradigm of modeling the non-nested overlaps with the rules left in $J$.

%\todo[for=Lincen]{
% Hmm.. if we continue iterating until no nested overlap remains, then this means that the two-rule nested overlap case discussed before can never occur when there are initially more than two rules covering an instance? Then the extension is not completely trivial, as in, it changes what actually happens compares to two overlapping rules?
%(Not sure whether it is clear now. See above.)}

\medskip \noindent
\textbf{Probabilistic rule sets.}
We can now build upon the previous to define rule sets as probabilistic models. 
% Formally, a probabilistic model is a family of probability distributions, with parameters to be estimated from the data, which becomes a fixed probability distribution with parameters fixed. 
Formally, the probabilistic model corresponding to a rule set $\ruleset$ is a family of probability distributions, denoted $P_{\ruleset, \theta}(Y|X)$ and parametrized by $\theta$. Specifically, $\theta$ is a parameter vector representing all necessary probabilities of $Y$ conditioned on events $\{X \in G\}$, where $G$ is either a single rule or the union of multiple rules. $\theta$ is estimated from data by estimating each $P(Y|X \in G)$ by its maximum likelihood estimator. The resulting estimated vector is denoted as $\htheta$ and contains $\hat{P}(Y|X \in G)$ for all ${G \in \mathscr{G}}$, where $\mathscr{G}$ consists of all individual rules and the unions of overlapping rules in $M$. 

Finally, we assume the dataset $D = (x^n, y^n)$ to be i.i.d. Specifically, let us define $(x,y) \vdash G$ for the following two cases: 1) when $G$ is a single rule (including the else rule), then $(x, y) \vdash G \iff x \in G$; and 2) when $G$ is a union of multiple rules, e.g., $G = \bigcup S_i$,  then $(x, y) \vdash G \iff x \in \bigcap S_i$. We then have
\begin{equation}
    P_{\ruleset, \theta}(y^n|x^n) = \prod_{G \in \mathscr{G}} \prod_{(x,y) \vdash G} P(Y = y | X \in G).
\end{equation}
% First, if a single rule (possibly the else-rule) covers $x'$, then this rule must have an associated multinomial model with estimated parameters and we use this multinomial distribution as $P(Y|X=x')$ and predict the most probable class label.

% Second, if $x'$ is covered by multiple rules, two possible cases exist: 1) if the disjunction of rules that covers $x'$ has an associated multinomial model with estimated parameters, we can use this multinomial distribution as $P(Y|X=x')$; 2) if the disjunction of rules that covers $x'$ has no associated multinomial distribution and parameter estimate, i.e., no training data point is covered by the overlap of these rules, a new multinomial model needs to be associated with this case and the parameter needs to be estimated from the training data. In practice, access to the training data is not necessary if the multinomial parameter and the number of covered data points are recorded for each rule: suppose $x'$ is covered by $S_i$ and $S_j$, then 1) if $S_i \subseteq S_j$ (or the reverse) then $\theta_{S_i \land S_j}$ must have already been estimated as all training data points in $S_i$ are also in $S_i \land S_j$; and 2) if $S_i$ does not fully cover $S_j$ (or the reverse) then the multinomial parameter for the case $x' \in S_i \land S_j$ can be calculated as the weighted sum $n_i \theta^*_i + n_j \theta^*_j$, where $n_i, n_j$ are the number of data points covered by $S_i, S_j$, respectively. 

%we use the maximum likelihood estimator $\theta_{S_{x'}}$ to calculate $P(y'|x')$ and predict the classification label of $x'$. Thus, with no separate conflicts resolving schemes, we guarantee that as long as the rule set has high predictive power, the approximation accuracy of each rule (or each union of rules for overlaps) is also high. Note that as union of rules will only exist when the associated probability distribution is not distinguishable from the probability distributions associated with all involved rules, high accuracy union of rules also indicates high accuracy of all involved rules. Hence, high quality rule set is guaranteed to have rules that are all high-quality. 











\section{Rule Set Learning as Probabilistic Model Selection}
\label{sec:model_selection}

Exploiting the formulation of rule sets as probabilistic models, we define the task of learning a rule set as a probabilistic model selection problem. Specifically, we use the minimum description length (MDL) principle for model selection. 

The MDL principle is one of the best off-the-shelf model selection methods and has been widely used in machine learning and data mining \citep{grunwald2019minimum, galbrun2022minimum}. Although rooted in information theory, it has been recently shown that MDL-based model selection can be regarded as an extension of Bayesian model selection \citep{grunwald2019minimum}. 

The principle of the MDL-based model selection is to pick the model, such that the code length (in bits) needed to encode the data given the model, together with the model itself, is minimized. We begin with discussing the Normalized Maximum Likelihood (NML) distributions for calculating the bits for encoding the data given the model, followed by the calculation of code length for encoding the model itself. 

\subsection{Normalized Maximum Likelihood Distributions for Rule Sets}
%\subsection{NML Distributions for Rule Sets}
%\label{subsec:nml_for_rule_sets}

%With its roots in information theory, the MDL principle aims at selecting the model that best compresses the data. Recent theoretic developments show that MDL-based model selection can be regarded as an extension of Bayesian model selection \citep{grunwald2019minimum}. 
As the Kraft inequality gaps the code length and probability, the core idea of the (modern) MDL principle it to assign a single probability distribution to the data given a rule set $\ruleset$ \citep{grunwald2019minimum}, the so-called \emph{universal distribution} denoted by $P_{\ruleset}(Y^n|X^n=x^n)$. Informally, $P_{\ruleset}(Y^n|X^n=x^n)$ should be a representative of the rule set model---as a family of probability distributions---$\{P_{\ruleset, \theta}(y^n | x^n)\}_\theta$.
The theoretically optimal ``representative" is defined to be the one that has minimax regret, i.e., 

\begin{equation}
\label{eq:define_regret}
	  \arg \min_{P_{\ruleset}} \max_{z^n \in \mathscr{Y}^n} -\log_2 P_{\ruleset} (Y^n = z^n|X^n = x^n) - \left(-\log_2 P_{\htheta(x^n, z^n)} (Y^n = z^n|X^n = x^n)\right).
\end{equation}
We write the parameter estimator as $\htheta(x^n, z^n)$ to emphasize that it depends on the values of the target variables $Y^n$. The unique solution to $P_{\ruleset}$ of Equation~\ref{eq:define_regret} is the so-called normalized maximum likelihood (NML) distribution:

\begin{equation}
    P^{NML}_{\ruleset}(Y^n=y^n|X^n=x^n) = \frac{P_{\ruleset, \htheta(x^n, y^n)}(Y^n=y^n|X^n=x^n)}{\sum_{z^n \in \mathscr{Y}^n} P_{\ruleset, \htheta(x^n, z^n)}(Y^n = z^n|X^n=x^n)}.
\end{equation}
That is, we ``normalize" the distribution $P_{\ruleset, \htheta}(.)$ to make it a proper probability distribution, which requires the sum of all possible values of $Y^n$ to be 1. Hence, we have $\sum_{z^n \in \mathscr{Y}^n} P^{NML}_{\ruleset}(Y^n=z^n|X^n=x^n) = 1$ \citep{grunwald2019minimum}.

\subsection{Approximating the NML Distribution}

A crucial difficulty in using the NML distribution in practice is the computation of the normalizing term $\sum_{z^n} P_{\htheta(x^n, z^n)}(Y^n=z^n|X^n=x^n)$. Efficient algorithms almost only exist for exponential family models \citep{grunwald2019minimum}, hence we approximate the term by the product of the normalizing terms for the individual rules. 

\smallskip
\noindent \textbf{NML distribution for a single rule.}
For an individual rule $S \in \ruleset$, we write all instances covered by $S$ as $(x^S, y^S)$, in which $y^S$ can be regarded as a realization of the random vector $Y^S = (Y, ..., Y)$, and $Y^S$ takes values in $\mathscr{Y}^{|S|}$, the $|S|$-ary Cartesian power of $\mathscr{Y}$. Then, the NML distribution for $P_S(Y)$ equals
\begin{equation}
    P^{NML}_S(Y^S = y^{S}|X^S = x^S) = \frac{\hat{P}_S(Y^S = y^S|X^S = x^S)}{\sum_{z^S \in \mathscr{Y}^S} \hat{P}_S(Y^S = z^S|X^S = x^S)}.
\end{equation}
Note that $\hat{P}_{S}$ depends on the values of $z^S$. As $\hat{P}_S(Y)$ is a categorical distribution, the normalizing term can be written as $\mathcal{R}(|S|, |\mathscr{Y}|)$, a function of $|S|$---the rule's coverage---and $|\mathscr{Y}|$---the number of unique values that $Y$ can take:
\begin{equation}
    \mathcal{R}(|S|, |\mathscr{Y}|) = \sum_{z^S \in \mathscr{Y}^S} \hat{P}_S(Y^S = z^S|X^S = x^S),
\end{equation}
and it can be efficiently calculated in sub-linear time~\citep{mononen:08:sub-lin-stoch-comp}.

\smallskip
\noindent \textbf{The approximate NML distribution.}
We propose to approximate the normalizing term of $P^{NML}_{\ruleset}$ as the product of the normalizing terms of $P^{NML}_S$ for all $S \in \ruleset$, and propose the approximate-NML distribution as our model selection criterion:
\begin{equation}
    P^{apprNML}_{\ruleset}(Y^n =y^n | X^n=x^n) = \frac{P_{\ruleset, \htheta(x^n, y^n)}(Y^n=y^n|X^n=x^n)}{\prod_{S \in \ruleset} \mathcal{R}(|S|, |\mathscr{Y}|)}.
\end{equation}
Note that the sum over all $S \in \ruleset$ \emph{does} include the ``else rule" $S_0$. 

The rationale of using the approximate-NML distribution is as follows. First, it is equal to the NML distribution for a rule set without any overlap, as follows.
\begin{proposition}
Given a rule set $\ruleset$ in which for any $S_i, S_j \in \ruleset$, $S_i \cap S_j = \emptyset$, then $P^{NML}_{\ruleset}(Y^n=y^n|X^n=x^n) = P^{apprNML}_{\ruleset}(Y^n=y^n|X^n=x^n)$.
\end{proposition}

\noindent Second, when overlaps exist in $\ruleset$, approximate-NML puts a small extra penalty on overlaps, which is desirable to trade-off overlap with goodness-of-fit: when we sum over all instances in each rule $S \in \ruleset$, the instances in overlaps are ``repeatedly counted". Third, approximate-NML behaves like the Bayesian information criterion (BIC) asymptotically, which follows from the next proposition.
\begin{proposition}
Assume $\ruleset$ contains $K$ rules in total, including the else rule, and we have $n$ instances. Under the mild assumption that $|S|$ grows linearly as the sample size $n$, then \\
$\log \left(\prod_{S \in \ruleset} \mathcal{R}(|S|, |\mathscr{Y}|)\right) = \frac{K(|\mathscr{Y}| - 1)}{2} \log n + \mathcal{O}(1)$, where $\mathcal{O}(1)$ is bounded by a constant w.r.t.\ to $n$.
\end{proposition}
We defer the proofs of the two propositions to the Appendix. 
% \begin{equation}
%       \sum_{y^n} P_{\theta^*(y^n|x^n)}(y^n|x^n) ={\sum_{G \in {2 ^ \ruleset}}} {\sum_{y^{|G|}}} P_{\theta_G} (y^{|G|}|x^{|G|})
% \end{equation}

 \subsection{Code length of model and final optimization score}
 To obtain the final MDL based score, we now describe how to calculate the code length of model $L(\ruleset) = -\log P(\ruleset) $.

 To encode the rule set, we encode each rule one at a time. Then, for each rule, we need to sequentially encode all of its conditions. Assume the features are $d$-dimensional, we firstly need to encode which dimension a rule condition, e.g., $X_j \in R_j$ is on, which costs $\log d$ bits. Next, we need to encode $R_j$ explicitly, which we discuss respectively for quantitative and categorical variables.

 If $X_j$ is categorical and has $c_j$ unique values, $R_j$ is a set containing a subset of all unique values. To encode $R_j$, we need to firstly encode the number of unique values of $X_j$ contains, denoted as $c_j'$, which costs $\log c_j$ bits; secondly, we need to encode what the $c_j'$ unique values are, which costs $\log {c_j \choose c_j'}$ bits. Thus, if we denote the bits needed in total to encode $X_j \in R_j$ as $l(X_j \in R_j)$, $l(X_j \in R_j) = \log d + \log c_j + \log {c_j \choose c_j'}$. 

 Then, if $X_j$ is quantitative, we firstly need to choose the candidate cut points for $X_j$. Common choices include equal binning cut and quantile cut into $m$ bins, where $m$ is a user-specified parameter. In practice, $m$ can be chose based on the computational budget and the granularity that is useful for the task at hand. To encode $X_j \in R_j$, we need to encode one or two cut points, respectively for the form $X_j \leq c$ (or `$>$') and $c_1 < X_j \leq c_2$. The cost of bits to encode an individual cut point $c$ is $\log {m-1 \choose c}$, and to encode `$\leq$' or `$>$' costs 1 bit. Thus, to encode $X_j \leq c$ costs $l(X_j \leq c) = \log d + \log {l-1 \choose c} + 1$ bits and the same for $X_j > c$, and to encode $c_1 < X_j \leq c_2$ costs $l(c_1< X_j \leq c_2) = 2(\log d + \log {l-1 \choose c} + 1)$ bits. 

 In summary, the code length needed to encode $\ruleset$ is 
 \begin{equation}
 	L(\ruleset) = \sum_{S \in \ruleset} L(S) = \sum_{S \in \ruleset} \sum_{k=1}^{len(S)} l(C_k) \, ,
 \end{equation}
 where $len(S)$ is the number of conditions in rule S, and $l(C_k)$ is the number of bits needed to encode the condition. 
 
Finally, we can formally define the optimal rule set $\ruleset^*$ as
\begin{equation}
    \ruleset^* = \arg \min_{\ruleset} -\log_2 P^{apprNML}_{\ruleset}(Y^n =y^n | X^n=x^n) + L(M).
\end{equation}
 
% Thus, the final MDL model selection score, denoted as $\mathcal{F}(y^n, \ruleset|x^n)$, is the sum of the minus-log-NML distribution and the code length of the model, i.e., 
% \begin{equation} \label{eq:MDL_score}
% 	\mathcal{F}(y^n, \ruleset|x^n) = -\log_2 P^{NML}_{\ruleset}(y^n|x^n) + L(\ruleset) \,.
% \end{equation}




\section{Learning Truly Unordered Rules from Data} \label{sec:alg}
Learning decision rules from data is an extremely difficult task; hence, rule learning algorithms have relied on heuristics that try to characterize the ``quality" of individual rules. While recently algorithms that can obtain provably optimal rule lists~\citep{angelino2017learning} and decision trees~\citep{hu2019optimal} are invented, their approach are not applicable to our task due to the following reasons. 

First, our model class (search space) is different as we allow for overlaps among rules, which brings unique challenges even for heuristic algorithms. Second, our model output is probabilistic while the optimal trees/lists algorithms learn rule-based models with non-probabilistic (or just binary) output. Third, our model selection criterion, while requiring no (regularization) hyper-parameters, does not allow efficient search for the global optimum, as like most existing MDL-based approaches~\citep{galbrun2022minimum}. Hence, we also cannot easily apply the branch-and-bounds approaches in the optimal trees/lists algorithms. 

We resort to heuristics as a result, and specifically, we iteratively add the next best rule to the rule set until the MDL-based score is optimized. 

\subsection{Compression ``learning" rate} \label{subsec:learning_rate}
When iteratively searching for the next ``best" rule, defining ``best'' is far from trivial: rule coverage and precision are contradicting factors and typical scores therefore combine those two factors in some---more or less---arbitrary way. 

This issue is further aggravated by the iterative rule learning process, in which the intermediate rule set is evaluated as an \emph{incomplete rule set} in each step. Evaluating incomplete rule sets is a challenging task~\citep{furnkranz2005roc}, mainly because any good score needs to simultaneously consider two aspects: 1) how well do all the rules currently in the rule set describe the already covered instances; and 2) what is the ``potential" for the uncovered instances, in the sense that how well can those uncovered instances be described by rules that might be added later?

To resolve this issue, we consider a greedy heuristic by considering the compression learning rate, i.e., by adding a given rule $S$ to the (potentially incomplete) rule set $\tilde{M}$, how much does the MDL-based score decrease \emph{per extra covered instance}. Formally, 
\begin{equation} \label{eq:learning_rate}
	r(S) = \frac{L(y^n|x^n, \tilde{M}) + L(\tilde{M}) - \left(L(y^n|x^n, \tilde{M}, S) + L(\tilde{M}, S) \right)}{|\tilde{M} \cup S| - |\tilde{M}|}, 
\end{equation}
in which $|\tilde{M}|$ and $|\tilde{M} \cup S|$ respectively denotes the coverage before and after adding $S$ to the (potentially incomplete) rule set $|\tilde{M}|$. 

With the heuristic, our rule learning algorithm becomes a simple iterative algorithm, in which the core is how to find the next best rule. We can keep adding the rule to the rule set as long as the rule has positive $r(S)$: since positive $r(S)$ is equivalent to improvement of the MDL-based score, we do not need separate stopping criterion. We next discuss in detail how to find the next rule in detail. 

\subsection{Finding next rule}
To avoid having to traverse all possible rules when searching for the rule to add to an incomplete rule set, we resort to a common heuristic: we start with an empty rule and gradually refine it by adding literals---also referred to as \emph{growing} a rule \citep{furnkranz2012foundations}. 

For growing the rule, we leverage three individual heuristics, all for resolving different issues. Specifically, we use a \emph{diverse} beam search approach: when growing a rule $S$, we keep the best $w$ rules, each of which has a diverse coverage. 
 Further, we identify a unique algorithm challenge coming along with allowing only ``good" overlaps, and we leverage an auxiliary beam together with a surrogate score to resolve it, which will be discussed in Section~\ref{subsec:auxiliary_beam}. 
 Last, we introduce a local constraint calculated based on the MDL principle, which try to characterize the ``potential" of the left-out part when growing the grow, as discussed in Section~\ref{subsec:local_constraint}. 
 
\subsubsection{Beam search with diverse coverage} \label{subsec:diverse_beam_search}

\noindent \textbf{Motivation.} While we aim to iteratively search for the rule with best the compression learning rate $r(S)$ (Equation~\ref{eq:learning_rate}), it may be too greedy to directly use $r(S)$ to search for the best literal for the next step growth. In other words, $r(S)$, as the heuristic to evaluate the quality of incomplete rule set (and hence also the quality of \emph{complete} rules), may not be suitable to be \emph{directly} used as a quality measure for incomplete rules. 

\noindent \textbf{Method.} Given a potentially incomplete rule $S$, we search all candidate rules $\{S'\}$ that can be obtained by adding a literal to $S$, excluding those (as will be discussed in the following subsection) not satisfying the \emph{local compression constraint}. 

Given a beam width $W$, we categorize all candidate rules $\{S'\}$ into $W$ subgroups according to their coverage: the $w$th subgroup is defined as: 
\begin{equation}
	\{S'\}_w = \{S' \in 	\{S'\}: \frac{|S'|}{|S|} \in \left[\frac{w-1}{W}, \frac{w}{W}\right) \}, \,\,\,\,\,\, w \in \{1, ..., W\}; 
\end{equation}
i.e., the coverage of all candidate growth results $S'$ in $\{S'\}_w$ divided by the coverage of $S$ must be in the interval $[(w-1)/W, w/W)$. 

Next, for each subgroup, we search for the best growth result by optimizing the compression learning rate $r(S')$. In this way, our beam search is diverse regarding the degree of ``patience": when the coverage decreases by a small ratio only, the optimization is ``patient" (by leaving a lot of possibilities for adding more literals); on the other hand, when the coverage decreases by a large ratio, the optimization is greedy (by leaving out little room for further refinement). 

\subsubsection{Auxiliary beam with a surrogate score} \label{subsec:auxiliary_beam}
\begin{figure}[ht]
    \centering
    \includegraphics[width=0.4\textwidth, height=0.3\textwidth]{alg1_1_new}
    \includegraphics[width=0.4\textwidth, height=0.3\textwidth]{alg1_2_new}
    \caption{(Left) Simulated data with two overlapping rules: $S_1: X_1 < 0.5$ (outlined in black) and $S_2: 0.5 < X_2 < 1$ (purple). (Right) $S_2$ has  grown to $0.5 < X_2 < 1 \land X_1 < 1.8$, which changes $P(Y|X \in S_2)$ and  resolves the problematic overlap.
    }
    \label{fig:alg1}
\end{figure}
\begin{figure}[ht]
    \centering
    \includegraphics[width=0.4\textwidth, height=0.3\textwidth]{alg2_1}
    \includegraphics[width=0.4\textwidth, height=0.3\textwidth]{alg2_2}
    \caption{(Left) Simulated data with a rule set containing two rules (black outlines). (Right) Growing a rule to describe the bottom-right instances will create conflicts with existing rules. I.e., adding either $X_1 > 1$ (vertical purple line) or $X_2 < 0.8$ (horizontal purple line) would create a huge overlap that deteriorates the likelihood.}
    \label{fig:alg2}
\end{figure}
Our another novel algorithmic invention is to use an additional auxiliary beam, which is designed to solve the specific algorithmic challenge coming along with our rule set model that only allows overlaps formed by rules with similar probabilistic outputs. 

\noindent \textbf{Motivation.} A rule can only has positive compression learning rate---and thus improve the MDL-based model selection score when it is added to the rule set---if it achieves two goals: 1) it should improve the likelihood of currently uncovered instances (penalized by the approximate-NML normalizing term); and 2) it should \emph{not} deteriorate the goodness-of-fit of the rule set by creating ``bad'' overlaps. These goals can be conflicting though, for two reasons.
   
First, it is not necessarily bad to have overlaps between a rule being grown and the current rule set, because the rule and its probability estimates for the target variable may still change. For example, consider the left plot of Figure~\ref{fig:alg1}. If the current rule set consists of $S_1$ (indicated in black), then adding $S_2$ (in purple) would be problematic: this would strongly deteriorate the likelihood of the instances covered by both rules. However, as we further grow $S_2$, as shown in the right plot, we get $P(Y|S_1) = P(Y|S_2)$ and the problem is solved.

Second, rules already in the rule set may become ``obstacles" to growing a new rule. For example, consider the data and rule set with two rules (in black) in Figure~\ref{fig:alg2}. If we want to grow a rule that covers the bottom-right instances, the existing rules form a blockade: the right plot shows how adding either $X_1 > 1$ or $X_2 < 0.8$ to the empty rule (in purple) would create a large overlap with the existing rules, with significantly different probability estimates.

\noindent \textbf{Surrogate score.}
We propose to use an auxiliary beam together with a surrogate score which informally is the compression learning rate when \emph{ignoring the overlap of between the (potentially incomplete) rule set and the currently growing rule}. 

Formally, given an the ruleset $\tilde{M}$ and a rule $S$, the surrogate score after adding $S$ to $\tilde{M}$, denoted as $R(S)$, is calculated as follows: 1) the probability estimates and likelihood for instances covered by $\tilde{M}$ remains unchanged; 2) the probability estimates, likelihood, and the NML distribution for instances covered by $S$ but \emph{excluding} those covered by $\tilde{M}$, denoted as $S \setminus \tilde{M}$, are calculated based on the ML estimator from $S \setminus \tilde{M}$ while ignoring the overlap between $S$ and $\tilde{M}$. Thus, the surrogate score is defined as 
\begin{equation} \label{eq:surrogate_learning_rate}
	R(S) = \frac{-\log_2 P_M^{apprNML}(Y^{S \setminus \tilde{M}} = y^{S \setminus \tilde{M}} | X^{S \setminus \tilde{M}} = x^{S \setminus \tilde{M}}) + L(S)} {|S \setminus \tilde{M}|},
\end{equation}
in which the numerator can be viewed as the change of the MDL-based score when adding rule $S$ to the rule set $\tilde{M}$ by assuming that $S$ only covers $S \setminus \tilde{M}$. 

By ``ignoring" the overlaps, our surrogate score can search for rule growth candidates that ignores the ``obstacles" from overlaps for now, which create potentially good candidates in the rule growth process. 

\subsubsection{Grow rules with the local compression constraint} \label{subsec:local_constraint}
When growing a rule $S$ by adding a literal and obtaining its growth result $S'$, we essentially leave out part of the cover of $S$ to be potentially covered by future rules. Existing rule learning heuristics often neglect this left-out part but focus only on characterizing the quality of the rule itself. In contrast, we introduce a local constraint that can be viewed as a way of approximately testing whether it is better keep the left-out part, or to leave it for later rules. 

Formally, consider a rule $S$, its growth result $S'$, and the left-out part $S \setminus S' := S_l$, we test whether 
\begin{equation} \label{eq:local_constraint}
	-\log_2 P_S^{apprNML}(y^S|x^S) > -\log_2 P_{S'}^{apprNML}(y^{S'}|x^{S'}) - \log_2 P_{S_l}^{apprNML}(y^{S_l}|x^{S_l}) + L_{split}, 
\end{equation}
where $L_{split}$ is the code length needed to encode the condition that splits $S$ into $S'$ and $S_l$. This requires specifying the variable of the literal and the numeric threshold or the categorical levels (which depends on the variable type). 

Intuitively, this is equivalent to building a depth-one decision tree for all instances covered by $S$, in which the left and right nodes are $S'$ and $S_l$. We then test whether the cover of $S$ can be better compressed by splitting $S$ into its two children, one being its growth result and the other being the left-out part. together with the code length needed to encode the split condition. 

Hence, our additional heuristic is to only allow the growth that satisfy the constraint defined in Equation~\ref{eq:local_constraint}. The rationale is that, if the left-out part can lead to a reasonable compression of its cover when it is reviewed as a separate rule (tree node), it is likely to find some other rules easily later to cover the left-out part, potentially together with other uncovered instances or just partially. 

Finally, note that we use the local constraints both for the beam and the auxiliary beam: when \emph{not} ignoring the overlap, we consider all instances covered by the rule $S$; when ignoring the overlap, we consider instances covered by the rule $S$ minus those covered by the (potentially incomplete) ruleset. 

 
%\begin{algorithm}[h] 
%\DontPrintSemicolon
%% \tcp{CUTS: candidate cuts for all dimensions}
%  \KwInput{(Incomplete) rule set $\tilde{M}$, data $(x^n, y^n)$}
%  \KwOutput{The next best rule $S^*$}
%%  \textbf{INPUT:} RULESET, $(x^n, y^n)$;   \textbf{OUTPUT:} RULE; 
%  
%  RULE $\assign$ $\emptyset$; Beam $\assign$ [RULE] \tcp*{Initialize the empty rule and beam}
%%  \tcp*{initialized as an empty array}
%  
%%  \tcp*{initial beam}
%  
%  BeamList $\assign$ Beam \tcp*{Record all the beams in the beam search}
%  
%%  $w = 5$ \tcp*{beam width}
%%  \While{TRUE}
%\While{$\operatorname{length}(Beam)  \neq 0$}
%  {
%    candidates $\assign$ [ ] \tcp*{initialized to store all possible refinements}
%
%    \For {RULE $\in$ Beam}     
%    {
%        Rs $\assign$ [Append L to RULE for L $\in$ all possible literals] 
%        
%        candidates.extend(Rs)
%    }
%    
%    Beam $\assign$ the $w$ rules in candidates that have 1) the highest positive $g_{unc}()$, and 2) coverage diversity $> \alpha$ \tcp*{$w$ is the beam width}
%    
%    \If {$\operatorname{length}$(Beam) $ \neq 0$}
%    {    
%        BeamList.extend(Beam) \tcp*{extend the BeamList as an array}
%    }
%    
%  }
%  
%  \For{Rule $\in$ BeamList}
%  {
%    Beam $\assign$ $w$ rules in BeamList with best $L_{\mathcal{T}}(\ruleset \oplus S_{unc})$ 
%  }
% \Return{Beam}
%\caption{Find Next Rule}
%\label{alg:find_next_rule}
%\end{algorithm}	

\begin{algorithm}[ht]
\caption{Find Next Rule} \label{alg:find_next_rule}
\begin{algorithmic}[1]
\REQUIRE (Incomplete) rule set $\tilde{M}$, dataset $(x^n, y^n)$, beam width $W$;
\ENSURE The next rule $S$, potentially to be added to $\tilde{M}$;
\STATE $\text{All\_candidate\_rules} \leftarrow \text{Empty Array}$
\STATE $\text{rules\_for\_next\_iter} \leftarrow \text{[Empty Rule]}$ \COMMENT{Initialize the rule with an ``empty" condition}
\WHILE{TRUE}
	\STATE $\text{beam} \leftarrow \text{EmptyBeam}()$ \COMMENT{Initialize the beam for the beam search}
	\STATE $\text{auxiliary\_beam} \leftarrow \text{EmptyBeam}()$ \COMMENT{Initialize the auxiliary beam  (Section~\ref{subsec:auxiliary_beam}})
	\FOR{$\text{rule}$ in $\text{rules\_for\_next\_iter}$}
		\STATE $\text{rule\_candidates} \leftarrow \text{generate\_candidates}(\text{rule})$
		\STATE $\text{categorized\_candidates} \leftarrow \text{categorize}(\text{rule\_candidates})$ \COMMENT{categorize candidates into subgroups by their coverage (Section~\ref{subsec:diverse_beam_search})}
		\STATE $\text{best\_W\_candidates} \leftarrow \text{candidate} \in \text{categorized\_candidates satisfying the }$ 
		\STATE $\,\,\,\,\,\,\,\,\,\,\,\,\,\,\,\, \text{ \textbf{local constraint} with the best }  r(candidate) \text{ in each category}$ 
		\COMMENT{local constraint is defined in Section~\ref{subsec:local_constraint}; r(.) defined in Equation~\ref{eq:learning_rate}}
		
		\STATE $\text{categorized\_candidates\_auxiliary} \leftarrow \text{categorize}(\text{rule\_candidates})$ \COMMENT{categorize candidates into subgroups by their coverage excluding the instances covered by $\title{M}$ (Section~\ref{subsec:diverse_beam_search})}
		\STATE $\text{best\_W\_auxiliary} \leftarrow \text{ candidate} \in \text{categorized\_candidates\_auxiliary satisfying}$
		\STATE $\,\,\,\,\,\,\,\,\,\,\,\,\,\,\,\, \text{ \textbf{local constraint} with the best }  R(candidate) \text{ in each category}$  \COMMENT{R(.) defined in Equation~\ref{eq:surrogate_learning_rate}}
		\STATE $\text{update\_beam} (\text{beam}, \text{best\_W\_candidates})$ \COMMENT{update using r(.)}
		\STATE $\text{update\_beam} (\text{auxiliary\_beam}, \text{best\_W\_auxiliary})$ \COMMENT{update using R(.)}
	\ENDFOR
	\STATE $\text{stop} \leftarrow \text{check\_whether\_stop}()$
	\IF{$\text{stop}$}
		\STATE $\text{among all } S \in \text{All\_candidate\_rules}$
		\RETURN \text{the S with the largest } r(S)
	\ELSE
		\STATE $\text{All\_candidate\_rules.append(beam)}$
		\STATE $\text{rules\_for\_next\_iter} \leftarrow \text{beam} \cup \text{auxiliary\_beam}$
	\ENDIF
\ENDWHILE
\end{algorithmic}
\end{algorithm}


\subsubsection{Algorithm description}
 After we detailedly discuss the heuristics, we now put all heuristics together and describe in full our algorithm for finding the next rule, of which the pseudo code is provided in Algorithm~\ref{alg:find_next_rule}. 

Assume the potentially empty and potentially incomplete ruleset is $\tilde{M}$, we always start with an \emph{Empty Rule} which contains no literals for its condition, and we initilize the ``rules\_for\_next\_iter" \textbf{[Line 2]} as an array containing the Empty Rule only. 

For each iteration, we initilize a new beam an a new auxiliary beam \textbf{[Line 4-5]} with beam width $W$. The beam keeps the $W$ best rule growth results using the criterion ``compression learning rate" defined in Equation~\ref{eq:learning_rate}; in contrast, the auxiliary beam keeps the $W$ best rule growth results using the ``surrogate" score by ignoring the rule's overlap with $\tilde{M}$, as discussed in detail in Section~\ref{subsec:auxiliary_beam}. 

Next, we use every rule in the ``rules\_for\_next\_iter" array as a ``base" for growing \textbf{[Line 6-16]}. Specifically, given a rule, we first generate its candidate growth \emph{by adding one  literal only} \textbf{[Line 6]}. That is, we go over all feature variables in the dataset, and for each variable, we generate candidate literals with numeric thresholds (often quantiles) or with categorical levels, based on the variable type. For numeric variables, we need to specify the granularity of search, which can be considered as a hyper-parameter. 

Further, we categorize the candidates by their coverage (for the beam) and their coverage excluding the instances already covered by $\tilde{M}$ (for the auxiliary beam) \textbf{[Line 8 \& 11]}. We then search for the best candidate in the beam and the auxiliary beam, using $r(.)$ and $R(.)$ (Section~\ref{subsec:auxiliary_beam}) as the criterion respectively, together with the local constraint (Section~\ref{subsec:local_constraint}) \textbf{[Line 9-10 \& 11-12]}. Note that the ``diverse coverage heuristic" is only used for growing individual rules; however, when updating ``beam" and ``auxiliary\_beam", we do not take into consideration the coverage heuristic, as the growth candidates from different rules cannot be categorized \textbf{[Line 14-15]}. 

To check whether the growing process should be stopped after this iteration, a greedy approach would be to stop when the ``beam" of this iteration does not produces rules with better compression learning rate $r(S)$ than the previous iteration's ``beam". Yet, we took a less greedy approach that we only stop when this is $K_{stop}$-th time in a row that both beams (i.e., the beam and the auxiliary beam) produce ``worse" rules than the previous beams. Note that the criterion of being ``worse" for the auxiliary beam is the surrogate score \textbf{[Line 17]}. Specifically, $K_{stop}$ is a user-defined parameter that controls the ``budget" of the algorithm. In practice, we find $K_{stop} = 5$ sufficient. 

Last, if the stopping criterion is not met, we update ``All\_candidate\_rules" and \\
``rules\_for\_next\_iter" \textbf{[Line 22-23]}, and continue to the next iteration \textbf{[Line 3]}. The former is the pool we use for finally selecting the next best rule to be potentially added to $\tilde{M}$. On the other hand, if the stopping criterion is met, we return the rule $S$ among ``All\_candidate\_rules" with the best (largest) compression learning rate $r(.)$ \textbf{[Line 19-20]}. 








% latex table generated in R 4.2.1 by xtable 1.8-4 package
% Thu Aug 17 16:32:39 2023
\begin{table}[ht]
\centering
\begin{tabular}{|l|ll|l|cc|}
  \hline
Data & \# rows & \# columns & \# classes & max. class prob. & min. class prob. \\ 
  \hline
aloi & 49534 &   28 &    2 & 0.970 & 0.030 \\ 
  backdoor & 95329 &  197 &    2 & 0.976 & 0.024 \\ 
  backnote & 1372 &    5 &    2 & 0.555 & 0.445 \\ 
  chess & 3196 &   37 &    2 & 0.522 & 0.478 \\ 
  diabetes &  768 &    9 &    2 & 0.651 & 0.349 \\ 
  glass-2 &  214 &    8 &    2 & 0.958 & 0.042 \\ 
  ionosphere &  351 &   35 &    2 & 0.641 & 0.359 \\ 
  magic & 19020 &   11 &    2 & 0.648 & 0.352 \\ 
  mammography & 11183 &    7 &    2 & 0.977 & 0.023 \\ 
  musk & 3062 &  167 &    2 & 0.968 & 0.032 \\ 
  optdigits & 5216 &   65 &    2 & 0.971 & 0.029 \\ 
  pendigits-2 & 6870 &   17 &    2 & 0.977 & 0.023 \\ 
  satimage-2 & 5803 &   37 &    2 & 0.988 & 0.012 \\ 
  smtp & 95156 &    4 &    2 & 1.000 & 0.000 \\ 
  thyroid & 3772 &    7 &    2 & 0.975 & 0.025 \\ 
  tic-tac-toe &  958 &   10 &    2 & 0.653 & 0.347 \\ 
  vowels & 1456 &   13 &    2 & 0.966 & 0.034 \\ 
  waveform-2 & 3443 &   22 &    2 & 0.971 & 0.029 \\ 
  wdbc &  367 &   31 &    2 & 0.973 & 0.027 \\ 
  \hline
  
  anuran & 7195 &   24 &    4 & 0.614 & 0.009 \\ 
  avila & 20867 &   11 &   12 & 0.411 & 0.001 \\ 
  car & 1728 &    7 &    4 & 0.700 & 0.038 \\ 
  contracept & 1473 &   10 &    3 & 0.427 & 0.226 \\ 
  drybeans & 13611 &   17 &    7 & 0.261 & 0.038 \\ 
  glass &  214 &   11 &    6 & 0.355 & 0.042 \\ 
  heartcleveland &  303 &   14 &    5 & 0.541 & 0.043 \\ 
  iris &  150 &    5 &    3 & 0.333 & 0.333 \\ 
  pendigits & 7494 &   17 &   10 & 0.104 & 0.096 \\ 
  vehicle &  846 &   19 &    4 & 0.258 & 0.235 \\ 
  waveform & 5000 &   22 &    3 & 0.339 & 0.329 \\ 
  wine &  178 &   14 &    3 & 0.399 & 0.270 \\ 
   \hline
\end{tabular}
\caption{Datasets for binary and multi-class classification. We use the maximum and minimum of the marginal class probabilities to indicate the degree of class imbalance. } 
\label{table:data_info}
\end{table} 
\section{Experiments} \label{sec:exp}
We benchmark our algorithm TURS on real-world datasets to empirically study the following research questions: 
\begin{enumerate}
     \vspace{-0.2cm}\item Can TURS achieve competitive classification performance with our carefully designed heuristics?
     \vspace{-0.2cm}\item Can TURS learn rules that can be empirically treated as \emph{truly unordered}? 
     \vspace{-0.2cm}\item Can TURS learn rules with class probability estimates that generalizes well to unseen (test) instances? That is, are the class probability estimates from the rules induced by TURS trustworthy for domain experts?
     \vspace{-0.2cm}\item What is the model complexity of the rule set model induced by TURS, given that interpretability is the core advantage of rule-based models?
\end{enumerate}

\subsection{Setup}
\textbf{Datasets.} We conduct an extensive experiments with 31 datasets, summarized in Table~\ref{table:data_info}. Our multi-class datasets are from the UCI repository, while the binary-class datasets are from both the UCI repository and the ADBench Github repository \citep{han2022adbench}. The latter is a benchmark toolbox for anomaly detection (including imbalanced classification), hence we include it to have both balanced and imbalanced binary datasets.

\noindent
\textbf{Competitors.} We compare against a wide ranges of methods: 1) unordered CN2, the one-versus-rest rule sets method without implicit order among rules; 2) DRS, a representative multi-class rule set method which aims for minimizing the size of overlaps; 3) IDS, the multi-class rule set method optimizing a linear combination of scores, which characterize the quality of rules in seven perspectives, 4) RIPPER, the widely used one-versus-rest method with orders among class labels, 5) CLASSY, the probabilistic rule list methods using MDL-based model selection; 6) CART, the well-known decision tree method, with post-pruning by a validation dataset separated from the training set; 7) C4.5 decision tree, with post-pruning by the MDL principle; 8) BRS, the bayesian rule sets method (only for binary dataset). 

\noindent{\textbf{Algorithm configurations.}} For TURS, we set the beam width as 10, and the number of candidate cut points for numeric features as 20. For competitor algorithms, we use CN2 from Orange~\citep{JMLR:demsar13a}, IDS from a third-party implementation with proved scalability \citep{filip2019pyids}, RIPPER and C4.5 from Weka~\citep{hall2009weka} and its R wrapper, CART from Python's Scikit-Learn package~\citep{scikit-learn}, and finally, DRS, BRS, CLASSY from the original authors. Competitors algorithms' configurations are set to be the same as the default as in the paper and/or in original authors' implementations. We make the code public for reproducibility, including the code for competitor algorithms\footnote{\url{...}}.


\begin{table}[ht]
\small
\centering
\begin{tabular}{l|llllllll|l}
  \hline
data & brs & c45 & cart & classy & ripper & cn2 & drs & ids & turs \tiny{(diff to best)} \\ 
%  \hline
%aloi & 0.519 & 0.398 & 0.621 & \textbf{0.654} & 0.485 & 0.569 & 0.500 & --- & 0.619\tiny{(-0.035)} \\ 
%  backdoor & 0.917 & 0.990 & 0.979 & 0.996 & 0.976 & \textbf{0.997} & --- & --- & 0.995\tiny{(-0.002)} \\ 
%  backnote & 0.957 & 0.987 & 0.983 & 0.990 & 0.982 & \textbf{0.993} & 0.988 & 0.505 & 0.981\tiny{(-0.012)} \\ 
%  chess & 0.957 & \textbf{0.998} & 0.995 & 0.992 & 0.995 & 0.532 & 0.809 & 0.708 & 0.994\tiny{(-0.004)} \\ 
%  diabetes & 0.725 & 0.710 & 0.667 & 0.737 & 0.641 & 0.709 & 0.727 & 0.528 & \textbf{0.750} \\ 
%  glass-2 & 0.676 & 0.890 & 0.790 & 0.730 & 0.793 & 0.941 & 0.926 & --- & \textbf{0.949} \\ 
%  ionosphere & 0.802 & 0.882 & 0.851 & 0.886 & 0.911 & \textbf{0.941} & 0.712 & 0.765 & 0.904\tiny{(-0.037)} \\ 
%  magic & 0.767 & 0.869 & 0.799 & \textbf{0.888} & 0.819 & 0.698 & 0.774 & 0.502 & 0.887\tiny{(-0.001)} \\ 
%  mammography & 0.644 & 0.817 & 0.730 & 0.890 & 0.582 & 0.891 & 0.857 & --- & \textbf{0.897} \\ 
%  musk & \textbf{1.000} & 0.995 & \textbf{1.000} & \textbf{1.000} & \textbf{1.000} & \textbf{1.000} & --- & --- & \textbf{1.000} \\ 
%  optdigits & 0.897 & 0.959 & 0.942 & 0.986 & 0.966 & \textbf{0.992} & --- & --- & 0.977\tiny{(-0.015)} \\ 
%  pendigits-2 & 0.938 & 0.986 & 0.964 & 0.974 & 0.973 & \textbf{0.996} & 0.948 & --- & 0.955\tiny{(-0.041)} \\ 
%  satimage-2 & 0.922 & 0.914 & 0.915 & 0.929 & \textbf{0.964} & \textbf{0.964} & 0.699 & --- & 0.909\tiny{(-0.055)} \\ 
%  smtp & 0.596 & 0.930 & 0.965 & 0.905 & 0.950 & 0.853 & 0.889 & --- & \textbf{0.972} \\ 
%  thyroid & 0.897 & 0.972 & 0.950 & 0.983 & 0.989 & \textbf{0.998} & 0.921 & --- & 0.961\tiny{(-0.037)} \\ 
%  tic-tac-toe & \textbf{1.000} & 0.878 & 0.918 & 0.978 & 0.972 & 0.932 & 0.992 & 0.782 & 0.965\tiny{(-0.035)} \\ 
%  vowels & 0.854 & 0.693 & 0.773 & 0.796 & 0.758 & \textbf{0.897} & 0.813 & --- & 0.817\tiny{(-0.08)} \\ 
%  waveform-2 & 0.567 & 0.716 & 0.648 & 0.847 & 0.333 & \textbf{0.886} & 0.540 & --- & 0.832\tiny{(-0.054)} \\ 
%  wdbc & 0.836 & \textbf{0.999} & 0.896 & 0.843 & 0.899 & 0.836 & 0.620 & --- & 0.947\tiny{(-0.052)} \\ 
%  anuran & --- & 0.995 & 0.944 & 0.968 & \textbf{0.996} & 0.962 & 0.945 & --- & 0.973\tiny{(-0.023)} \\ 
%  \hline
%  
%  avila & --- & \textbf{0.999} & 0.977 & 0.987 & 0.993 & 0.920 & 0.729 & 0.537 & 0.990\tiny{(-0.009)} \\ 
%  car & --- & 0.956 & 0.939 & 0.978 & 0.931 & 0.885 & 0.935 & 0.800 & \textbf{0.980} \\ 
%  contracept & --- & \textbf{0.680} & 0.597 & 0.653 & 0.607 & 0.598 & 0.598 & 0.542 & 0.658\tiny{(-0.022)} \\ 
%  drybeans & --- & 0.970 & 0.943 & 0.977 & 0.979 & 0.929 & 0.975 & --- & \textbf{0.989} \\ 
%  glass & --- & 0.970 & \textbf{0.984} & 0.975 & 0.940 & 0.937 & 0.926 & 0.533 & 0.967\tiny{(-0.017)} \\ 
%  heartcleveland & --- & 0.603 & 0.572 & \textbf{0.721} & 0.509 & 0.694 & 0.611 & 0.549 & 0.695\tiny{(-0.026)} \\ 
%  iris & --- & 0.960 & 0.975 & 0.970 & 0.962 & 0.977 & 0.954 & 0.762 & \textbf{0.981} \\ 
%  pendigits & --- & 0.982 & 0.974 & 0.986 & 0.983 & 0.991 & 0.967 & 0.567 & \textbf{0.994} \\ 
%  vehicle & --- & 0.856 & 0.789 & 0.870 & 0.859 & 0.858 & 0.764 & 0.575 & \textbf{0.882} \\ 
%  waveform & --- & 0.842 & 0.814 & 0.910 & 0.880 & 0.803 & 0.654 & 0.506 & \textbf{0.915} \\ 
%  wine & --- & 0.937 & 0.906 & 0.960 & 0.937 & \textbf{0.973} & 0.909 & 0.564 & 0.952\tiny{(-0.021)} \\ 
   \hline
aloi & 0.519 & 0.398 & 0.621 & \textbf{0.654} & 0.485 & 0.569 & 0.500 & 0.509 & 0.619 \tiny{(-0.035)} \\ 
  backdoor & 0.917 & 0.990 & 0.979 & 0.996 & 0.976 & \textbf{0.997} & --- & --- & 0.995 \tiny{(-0.002)} \\ 
  backnote & 0.957 & 0.987 & 0.983 & 0.990 & 0.982 & \textbf{0.993} & 0.988 & 0.765 & 0.981 \tiny{(-0.012)} \\ 
  chess & 0.957 & \textbf{0.998} & 0.995 & 0.992 & 0.995 & 0.532 & 0.809 & 0.677 & 0.994 \tiny{(-0.004)} \\ 
  diabetes & 0.725 & 0.710 & 0.667 & 0.737 & 0.641 & 0.709 & 0.727 & 0.595 & \textbf{0.750} \\ 
  glass-2 & 0.676 & 0.890 & 0.790 & 0.730 & 0.793 & 0.941 & 0.926 & 0.912 & \textbf{0.949} \\ 
  ionosphere & 0.802 & 0.882 & 0.851 & 0.886 & 0.911 & \textbf{0.941} & 0.712 & 0.786 & 0.904 \tiny{(-0.037)} \\ 
  magic & 0.767 & 0.869 & 0.799 & \textbf{0.888} & 0.819 & 0.698 & 0.774 & 0.507 & 0.887 \tiny{(-0.001)} \\ 
  mammography & 0.644 & 0.817 & 0.730 & 0.890 & 0.582 & 0.891 & 0.857 & 0.535 & \textbf{0.897} \\ 
  musk & \textbf{1.000} & 0.995 & \textbf{1.000} & \textbf{1.000} & \textbf{1.000} & \textbf{1.000} & --- & \textbf{1.000} & \textbf{1.000} \\ 
  optdigits & 0.897 & 0.959 & 0.942 & 0.986 & 0.966 & \textbf{0.992} & --- & 0.960 & 0.977 \tiny{(-0.015)} \\ 
  pendigits-2 & 0.938 & 0.986 & 0.964 & 0.974 & 0.973 & \textbf{0.996} & 0.948 & 0.914 & 0.955 \tiny{(-0.041)} \\ 
  satimage-2 & 0.922 & 0.914 & 0.915 & 0.929 & \textbf{0.964} & \textbf{0.964} & 0.699 & 0.867 & 0.909 \tiny{(-0.055)} \\ 
  smtp & 0.596 & 0.930 & 0.965 & 0.905 & 0.950 & 0.853 & 0.889 & 0.879 & \textbf{0.972} \\ 
  thyroid & 0.897 & 0.972 & 0.950 & 0.983 & 0.989 & \textbf{0.998} & 0.921 & 0.960 & 0.961 \tiny{(-0.037)} \\ 
  tic-tac-toe & \textbf{1.000} & 0.878 & 0.918 & 0.978 & 0.972 & 0.932 & 0.992 & 0.599 & 0.965 \tiny{(-0.035)} \\ 
  vowels & 0.854 & 0.693 & 0.773 & 0.796 & 0.758 & \textbf{0.897} & 0.813 & 0.748 & 0.817 \tiny{(-0.08)} \\ 
  waveform-2 & 0.567 & 0.716 & 0.648 & 0.847 & 0.333 & \textbf{0.886} & 0.540 & 0.774 & 0.832 \tiny{(-0.054)} \\ 
  wdbc & 0.836 & \textbf{0.999} & 0.896 & 0.843 & 0.899 & 0.836 & 0.620 & 0.942 & 0.947 \tiny{(-0.052)} \\ 
  \hline
  anuran & --- & 0.995 & 0.944 & 0.968 & \textbf{0.996} & 0.962 & 0.945 & 0.602 & 0.973 \tiny{(-0.023)} \\ 
  avila & --- & \textbf{0.999} & 0.977 & 0.987 & 0.993 & 0.920 & 0.729 & 0.617 & 0.990 \tiny{(-0.009)} \\ 
  car & --- & 0.956 & 0.939 & 0.978 & 0.931 & 0.885 & 0.935 & 0.831 & \textbf{0.980} \\ 
  contracept & --- & \textbf{0.680} & 0.597 & 0.653 & 0.607 & 0.598 & 0.598 & 0.549 & 0.658 \tiny{(-0.022)} \\ 
  drybeans & --- & 0.970 & 0.943 & 0.977 & 0.979 & 0.929 & 0.975 & 0.591 & \textbf{0.989} \\ 
  glass & --- & 0.970 & \textbf{0.984} & 0.975 & 0.940 & 0.937 & 0.926 & 0.793 & 0.967 \tiny{(-0.017)} \\ 
  heartcleveland & --- & 0.603 & 0.572 & \textbf{0.721} & 0.509 & 0.694 & 0.611 & 0.513 & 0.695 \tiny{(-0.026)} \\ 
  iris & --- & 0.960 & 0.975 & 0.970 & 0.962 & 0.977 & 0.954 & 0.810 & \textbf{0.981} \\ 
  pendigits & --- & 0.982 & 0.974 & 0.986 & 0.983 & 0.991 & 0.967 & 0.522 & \textbf{0.994} \\ 
  vehicle & --- & 0.856 & 0.789 & 0.870 & 0.859 & 0.858 & 0.764 & 0.579 & \textbf{0.882} \\ 
  waveform & --- & 0.842 & 0.814 & 0.910 & 0.880 & 0.803 & 0.654 & 0.517 & \textbf{0.915} \\ 
  wine & --- & 0.937 & 0.906 & 0.960 & 0.937 & \textbf{0.973} & 0.909 & 0.854 & 0.952 \tiny{(-0.021)} \\ 
   \hline
  

\end{tabular}
\caption{ROC-AUC scores on the test sets, averaged on five-fold stratified cross-validations.} \label{table:roc_auc}
\end{table}

\begin{figure}[ht]
\includegraphics[width=\textwidth]{boxplot_roc_auc}	
\caption{For each algorithm, we calculate for every individual dataset the difference between its ROC-AUC score and the best ROC-AUC scores. The differences to the best ROC-AUC scores for each algorithm is illustrated by a box-plot.} \label{fig:diff_to_best_auc}
\end{figure}

\subsection{Classification performance} \label{subsec:classifier_perf}
To investigate the classification performance for TURS, we report  in Table~\ref{table:roc_auc} the ROC-AUC scores on the test sets, averaged over five-fold cross-validations. For multi-class classification, we report the ``macro" one-versus-rest AUC scores, as ``macro" AUC treats all class labels equally and hence can characterize how well the classifiers predict for the minority classes. 

Note that BRS~\citep{wang2017bayesian} can only be applied to binary datasets. Further, we fail to obtain the results of DRS on three datasets because the implementation of DRS does not support the large number of columns (it simply gives an error). We also fail to obtain the results of IDS on several datasets as they exceed the predetermined time limit: 10 hours for one single fold of one dataset. 

We have shown that TURS is very competitive in comparison to its competitors in the following aspects. First, TURS performs the best in 11 out of the total 31 datasets, and performs the best in 6 out of 11 multi-class datasets. We denote the best ROC-AUC for each dataset in bold. 

Second, we report the difference between TURS's ROC-AUC scores and the best ROC-AUC scores for each individual dataset, in the bracket in the table. We also repeat this process for all competitor algorithms, obtaining their ``gaps-to-best" scores. We further demonstrate these gaps-to-best scores in Figure~\ref{fig:diff_to_best_auc}. The box-plots demonstrate that TURS is very stable for all 31 datasets we have tested: in comparison to its competitors, the gaps-to-best scores are much smaller. 

Third, among all rule sets methods (CN2, DRS, IDS, TURS), TURS show substantially superior performance against DRS and IDS. As DRS and IDS both aim to reduce the sizes of overlaps, our results indicate that minimizing the sizes of overlaps may impose a too restricted constraint and hence lead to sub-optimal classification performance. On the other hand, CN2 is competitive in terms of obtaining the best AUCs, especially for binary datasets, as shown in Table~\ref{table:roc_auc}. However,  as shown in Figure~\ref{fig:diff_to_best_auc}, CN2 has in general larger gaps to the best AUCs than TURS does. Further, more comparisons illustrating TURS superiorities over CN2 will be in the following paragraphs. 

\begin{table}[ht]
\centering
\small
\begin{tabular}{l|rrll|rrll}
  \hline
data & TURS & TURS-RP & Diff. & \%overlap & CN2 & CN2-RP & Diff. & \%overlap \\ 
  \hline
aloi & 0.6190 & 0.6200 & -0.001 & 4\% & 0.5690 & 0.5780 & -0.009 & 97\% \\ 
  backdoor & 0.9950 & 0.9950 & 0 & 0\% & 0.9970 & 0.9760 & 0.021 & 96\% \\ 
  backnote & 0.9810 & 0.9800 & 0.001 & 20\% & 0.9930 & 0.9730 & 0.02 & 60\% \\ 
  chess & 0.9940 & 0.9940 & 0 & 23\% & 0.5320 & 0.5510 & -0.019 & 95\% \\ 
  diabetes & 0.7500 & 0.7480 & 0.002 & 11\% & 0.7090 & 0.6760 & 0.033 & 82\% \\ 
  glass-2 & 0.9490 & 0.9490 & 0 & 0\% & 0.9410 & 0.8390 & 0.102 & 33\% \\ 
  ionosphere & 0.9040 & 0.9040 & 0 & 15\% & 0.9410 & 0.8950 & 0.046 & 55\% \\ 
  magic & 0.8870 & 0.8870 & 0 & 38\% & 0.6980 & 0.7380 & -0.04 & 92\% \\ 
  mammography & 0.8970 & 0.8970 & 0 & 5\% & 0.8910 & 0.8060 & 0.085 & 86\% \\ 
  musk & 1.0000 & 1.0000 & 0 & 0\% & 1.0000 & 1.0000 & 0 & 0\% \\ 
  optdigits & 0.9770 & 0.9770 & 0 & 0\% & 0.9920 & 0.9720 & 0.02 & 92\% \\ 
  pendigits-2 & 0.9550 & 0.9550 & 0 & 0\% & 0.9960 & 0.9720 & 0.024 & 88\% \\ 
  satimage-2 & 0.9090 & 0.9090 & 0 & 0\% & 0.9640 & 0.9090 & 0.055 & 89\% \\ 
  smtp & 0.9720 & 0.9720 & 0 & 0\% & 0.8530 & 0.7950 & 0.058 & 51\% \\ 
  thyroid & 0.9610 & 0.9610 & 0 & 0\% & 0.9980 & 0.9410 & 0.057 & 87\% \\ 
  tic-tac-toe & 0.9650 & 0.9650 & 0 & 7\% & 0.9320 & 0.9250 & 0.007 & 49\% \\ 
  vowels & 0.8170 & 0.8170 & 0 & 1\% & 0.8970 & 0.8380 & 0.059 & 71\% \\ 
  waveform-2 & 0.8320 & 0.8320 & 0 & 9\% & 0.8860 & 0.7540 & 0.132 & 92\% \\ 
  wdbc & 0.9470 & 0.9470 & 0 & 0\% & 0.8360 & 0.5960 & 0.24 & 69\% \\ 
  \hline
  anuran & 0.9730 & 0.9690 & 0.004 & 28\% & 0.9620 & 0.9130 & 0.049 & 90\% \\ 
  avila & 0.9900 & 0.9890 & 0.001 & 17\% & 0.9200 & 0.9150 & 0.005 & 45\% \\ 
  car & 0.9800 & 0.9800 & 0 & 22\% & 0.8850 & 0.7940 & 0.091 & 91\% \\ 
  contracept & 0.6580 & 0.6570 & 0.001 & 3\% & 0.5980 & 0.5720 & 0.026 & 100\% \\ 
  drybeans & 0.9890 & 0.9860 & 0.003 & 34\% & 0.9290 & 0.9080 & 0.021 & 94\% \\ 
  glass & 0.9670 & 0.9650 & 0.002 & 2\% & 0.9370 & 0.9370 & 0 & 0\% \\ 
  heartcleveland & 0.6950 & 0.6870 & 0.008 & 8\% & 0.6940 & 0.6630 & 0.031 & 61\% \\ 
  iris & 0.9810 & 0.9800 & 0.001 & 5\% & 0.9770 & 0.9770 & 0 & 0\% \\ 
  pendigits & 0.9940 & 0.9910 & 0.003 & 40\% & 0.9910 & 0.9820 & 0.009 & 76\% \\ 
  vehicle & 0.8820 & 0.8780 & 0.004 & 15\% & 0.8580 & 0.8260 & 0.032 & 77\% \\ 
  waveform & 0.9150 & 0.9050 & 0.01 & 51\% & 0.8030 & 0.8400 & -0.037 & 77\% \\ 
  wine & 0.9520 & 0.9520 & 0 & 0\% & 0.9730 & 0.9710 & 0.002 & 2\% \\ 
   \hline
\end{tabular}
\caption{ROC-AUC for the predictions with and without ``random picking", both for TURS and CN2, averaged by five-fold stratified cross-validations. For each fold, we repeat the random-picking predictions for 10 times, and report the average of all corresponding ROC-AUCs. } \label{table:roc_auc_rp}
\end{table}
\begin{figure}[ht] \centering
\includegraphics[width=0.75\textwidth]{boxplot_random_pick}	
\caption{Box-plots for ROC-AUC minus ROC-AUC with ``random picking", for TURS and CN2, averaged over the five-fold stratified cross-validations. } \label{fig:diff_rp}
\end{figure}

\subsection{Prediction with `random picking' for overlaps}
Recall that we estimate the class probabilities for overlaps by considering the ``union" of the cover of all involved rules. Thus, the next question we study empirically is whether our formalization of rule sets as probabilistic models can indeed lead to overlaps only formed by rules with similar probabilistic estimations. 

Therefore, we compare the (probabilistic) predictions of our TURS models against the probabilistic predictions by what we call ``random picking" for overlaps: when an unseen instance is covered by multiple rules, we randomly pick one individual rule, and use its estimated class probabilities (estimated from training set) as the probabilistic prediction for this instance. 

Intuitively, if the overlaps are formed only by rules with similar probabilistic output, we expect the probabilistic prediction performance by TURS and by TURS with random-picking (TURS-RP) to be very close. We first report the ROC-AUC of TURS and TURS-RP in Table~\ref{table:roc_auc_rp}, together with the percentage of instances covered by more than one rules (\%~overlaps), which we also benchmark against CN2 (IDS and DRS are excluded due to their sub-optimal performance in general). We observe that the ROC-AUC for TURS and TURS-RP is almost the same, while the ROC-AUC for CN2 and CN2 with random picking (CN2-RP) can differ substantially in general, as also summarized in Figure~\ref{fig:diff_rp}.
\begin{figure}[ht] \centering 
	\includegraphics[width=\textwidth]{logloss_randompicking}
	\caption{Log-loss of TURS, with and without the ``random picking", averaged over the stratified five-fold cross-validations. } \label{fig:logloss_rp}
\end{figure}

Further, we show in Figure~\ref{fig:logloss_rp} the log-loss of TURS and TURS-RP, to investigate the effect of the ``random picking" on the predictive class probabilities. We observe that, except for a few exceptions, the difference between the two log-losses for all datasets are minimal. Specifically, the few exceptions can be explained by 1) small sample sizes, including the dataset ``glass" (sample size 214), and ``vehicle" (sample size 846); 2) very large percentage of instances covered by multiple rules and including the dataset ``drybeans"~(34\%) and ``pendigits"~(40\%). 

We can hence conclude that, while CN2 relies heavily on its conflict resolving schemes for overlaps, TURS produces overlaps only formed by decision rules with very similar probability estimates. This indicates our decision rules can be viewed as \emph{truly unordered in the sense that, when an instance is covered by multiple rules, it has very little effect on which rule is picked to predict its class probabilities.} In other words, overlaps in our model provides a multi-perspectives descriptions to the corresponding instances; in contrast, all existing methods treat overlaps as ``nuisances". 

\begin{figure}[ht]\centering
\includegraphics[width=\textwidth]{train_test_diff_rulesets}
\includegraphics[width=\textwidth]{train_test_diff_treeandlist}
\caption{The weighted average of the differences between the class probability estimates of every individual rule for training and test sets, in which the weights the coverage of each rule (for the training set). The figures report the average on five-fold stratified cross-validations: the above figure shows the comparison against rule sets competitor methods, while the below one against decision list and tree methods.}
\label{fig:train_test_diff}
\end{figure}
\subsection{Generalizability of local probabilistic estimates}
While decision rules are widely accepted as intrinsically explainable models, rules with probability estimates that generalize well can serve as good explanations. Thus, we next examine the difference between individual rules' probability estimates on the train and test sets, with five-fold stratified cross-validations. 

Specifically, given a ruleset induced from a specific dataset, we look at each individual rule's probability estimates, estimated from the training and test set respectively, by the maximum likelihood estimator (MLE). We next take the average of the class probabilities (both for binary and multi-class targets). Finally, we report the weighted average for all rules, weighted by the coverage of each rule (on the training set). 

Formally, given a ruleset with $K$ rules: $M = \{S_1, ..., S_K\}$, denote the probability estimates of all rules by $(\mathbf{p}_1, ..., \mathbf{p}_K)$ and $(\mathbf{q}_1, ..., \mathbf{q}_K)$, respectively estimated from the training and test set. We measure how well the individual rules generalize by
\begin{equation} \label{eq:train_test_diff}
	g = \frac{1}{K}\sum_{j} |S_i| (\bar{\mathbf{p}_i} - \bar{\mathbf{q}_i}),
\end{equation}
in which $\bar{\mathbf{p}_i}$ and $\bar{\mathbf{q}_i}$ denote the mean of elements of the estimated class probability vectors. Note that each individual rule is treated separately in calculating the $g$-score above, and hence the overlaps do not play a role here. 

We calculate this score for all algorithms and all datasets, averaged over five-fold cross-validations, and demonstrate the results in Figure~\ref{fig:train_test_diff}. As a result, we observe that TURS (the bold purple curve) mostly lies at the lowest level, and is much more stable than all competitor algorithms. 

The only exception is IDS (blue line in the upper figure): while the rules' probability estimates of IDS generalize better than that of TURS in some datasets, this indicates IDS has serious ``under-fitting" if we take into consideration IDS's suboptimal predictive performance as discussed in Section~\ref{subsec:classifier_perf}. That is, IDS produces rules with too large coverage, and hence is not specific and refined enough for classification, although rules with large coverage have probability estimates that generalize well. 

%We also report the exact numbers for the score calculated by Equation~\ref{eq:train_test_diff} in Table~\ref{table:train_test_diff}.

Thus, in conclusion, rules induced by TURS serve better as explanations for the predictions of the tree- and rule-based models, in terms of the quality of its probability estimates. 

%% latex table generated in R 4.2.1 by xtable 1.8-4 package
% Thu Aug 17 17:08:45 2023
\begin{table}[ht]
\small
\centering
\begin{tabular}{l|llllllll|l}
  \hline
data\_name & brs & c45 & cart & classy & ripper & cn2 & drs & ids & turs \tiny{(diff to best)} \\ 
  \hline
aloi & 0.030 & 0.009 & 0.032 & 0.008 & 0.001 & 0.024 & 0.000 & 0.003 & 0.006 \tiny{(0.006)} \\ 
  backdoor & 0.003 & 0.001 & 0.001 & 0.017 & 0.001 & 0.002 & --- & --- & 0.001 \\ 
  backnote & 0.014 & 0.014 & 0.015 & 0.025 & 0.017 & 0.032 & 0.018 & 0.000 & 0.015 \tiny{(0.015)} \\ 
  chess & 0.006 & 0.005 & 0.006 & 0.030 & 0.007 & 0.221 & 0.015 & 0.000 & 0.005 \tiny{(0.005)} \\ 
  diabetes & 0.077 & 0.136 & 0.261 & 0.092 & 0.058 & 0.131 & 0.131 & 0.024 & 0.060 \tiny{(0.036)} \\ 
  glass-2 & 0.354 & 0.011 & 0.018 & 0.426 & 0.031 & 0.025 & 0.009 & 0.000 & 0.035 \tiny{(0.035)} \\ 
  ionosphere & 0.084 & 0.058 & 0.122 & 0.041 & 0.073 & 0.064 & 0.024 & 0.031 & 0.015 \\ 
  magic & 0.010 & 0.064 & 0.146 & 0.045 & 0.024 & 0.131 & 0.012 & 0.107 & 0.020 \tiny{(0.009)} \\ 
  mammography & 0.135 & 0.008 & 0.013 & 0.064 & 0.006 & 0.012 & 0.004 & 0.002 & 0.007 \tiny{(0.005)} \\ 
  musk & 0.000 & 0.000 & 0.000 & 0.010 & 0.000 & 0.002 & --- & 0.000 & 0.000 \\ 
  optdigits & 0.106 & 0.004 & 0.005 & 0.033 & 0.004 & 0.005 & --- & 0.000 & 0.004 \tiny{(0.004)} \\ 
  pendigits-2 & 0.038 & 0.002 & 0.002 & 0.027 & 0.002 & 0.003 & 0.001 & 0.000 & 0.003 \tiny{(0.002)} \\ 
  satimage-2 & 0.050 & 0.003 & 0.002 & 0.097 & 0.003 & 0.001 & 0.000 & 0.001 & 0.002 \tiny{(0.002)} \\ 
  smtp & 0.007 & 0.000 & 0.000 & 0.001 & 0.000 & 0.000 & 0.000 & 0.000 & 0.000 \tiny{(0)} \\ 
  thyroid & 0.076 & 0.003 & 0.002 & 0.118 & 0.003 & 0.002 & 0.013 & 0.000 & 0.005 \tiny{(0.005)} \\ 
  tic-tac-toe & 0.000 & 0.087 & 0.050 & 0.010 & 0.017 & 0.073 & 0.007 & 0.059 & 0.016 \tiny{(0.016)} \\ 
  vowels & 0.148 & 0.019 & 0.022 & 0.165 & 0.009 & 0.010 & 0.013 & 0.001 & 0.010 \tiny{(0.009)} \\ 
  waveform-2 & 0.315 & 0.019 & 0.030 & 0.085 & 0.018 & 0.012 & 0.303 & 0.001 & 0.013 \tiny{(0.011)} \\ 
  wdbc & 0.257 & 0.002 & 0.011 & 0.322 & 0.009 & 0.012 & 0.000 & 0.000 & 0.007 \tiny{(0.007)} \\ 
  anuran & --- & 0.002 & 0.015 & 0.016 & 0.001 & 0.028 & 0.005 & 0.003 & 0.009 \tiny{(0.008)} \\ 
  avila & --- & 0.001 & 0.003 & 0.007 & 0.011 & 0.019 & 0.008 & 0.005 & 0.002 \tiny{(0.002)} \\ 
  car & --- & 0.025 & 0.012 & 0.049 & 0.050 & 0.049 & 0.013 & 0.000 & 0.021 \tiny{(0.021)} \\ 
  contracept & --- & 0.166 & 0.296 & 0.055 & 0.042 & 0.117 & 0.044 & 0.004 & 0.036 \tiny{(0.032)} \\ 
  drybeans & --- & 0.015 & 0.023 & 0.010 & 0.018 & 0.043 & 0.007 & 0.000 & 0.009 \tiny{(0.009)} \\ 
  glass & --- & 0.007 & 0.006 & 0.023 & 0.015 & 0.037 & 0.023 & 0.006 & 0.027 \tiny{(0.021)} \\ 
  heartcleveland & --- & 0.109 & 0.161 & 0.098 & 0.032 & 0.054 & 0.118 & 0.167 & 0.059 \tiny{(0.028)} \\ 
  iris & --- & 0.041 & 0.019 & 0.010 & 0.046 & 0.043 & 0.033 & 0.009 & 0.034 \tiny{(0.025)} \\ 
  pendigits & --- & 0.006 & 0.007 & 0.007 & 0.028 & 0.015 & 0.005 & 0.004 & 0.005 \tiny{(0.001)} \\ 
  vehicle & --- & 0.087 & 0.126 & 0.064 & 0.084 & 0.076 & 0.105 & 0.010 & 0.034 \tiny{(0.024)} \\ 
  waveform & --- & 0.119 & 0.138 & 0.062 & 0.091 & 0.075 & 0.036 & 0.038 & 0.039 \tiny{(0.002)} \\ 
  wine & --- & 0.046 & 0.073 & 0.025 & 0.057 & 0.048 & 0.042 & 0.023 & 0.044 \tiny{(0.021)} \\ 
   \hline\end{tabular}
\caption{The weighted average of the differences between the class probability estimates of every individual rule for training and test sets, in which the weights the coverage of each rule (for the training set). The table reports the average on five-fold stratified cross-validations.} \label{table:train_test_diff}
\end{table}

% latex table generated in R 4.2.1 by xtable 1.8-4 package
% Thu Aug 17 16:54:14 2023
\begin{table}[ht]
\small
\centering
\begin{tabular}{llllllllll}
  \hline
data\_name & brs & c45 & cart & classy & ripper & cn2 & drs & ids & turs \\ 
  \hline
aloi & \scriptsize{3} & \scriptsize{2659} & 26953 & 52 & \scriptsize{26} & 2116 & --- & --- & 67 \\ 
  backdoor & \scriptsize{13} & 701 & 2460 & 73 & 102 & 260 & --- & --- & 59 \\ 
  backnote & 36 & 80 & 117 & 23 & 22 & 40 & 54 & \scriptsize{7} & 14 \\ 
  chess & 19 & 250 & 340 & 33 & 58 & \scriptsize{297} & \scriptsize{54} & \scriptsize{10} & 58 \\ 
  diabetes & 16 & 108 & \scriptsize{700} & \textbf{5} & \scriptsize{7} & 165 & 82 & \scriptsize{9} & 7 \\ 
  glass-2 & \scriptsize{11} & \scriptsize{20} & \scriptsize{7} & \scriptsize{3} & \scriptsize{6} & 2 & 37 & --- & \textbf{1} \\ 
  ionosphere & \scriptsize{31} & 59 & \scriptsize{87} & 6 & 13 & 25 & \scriptsize{441} & \scriptsize{11} & \textbf{5} \\ 
  magic & \scriptsize{44} & 3212 & \scriptsize{20053} & 228 & \scriptsize{87} & \scriptsize{3352} & \scriptsize{45} & \scriptsize{4} & 228 \\ 
  mammography & \scriptsize{3} & \scriptsize{274} & \scriptsize{1127} & 39 & \scriptsize{31} & 199 & 24 & --- & 37 \\ 
  musk & 6 & 4 & \textbf{2} & \textbf{2} & \textbf{2} & \textbf{2} & --- & --- & \textbf{2} \\ 
  optdigits & \scriptsize{55} & 47 & 79 & 10 & 16 & 17 & --- & --- & \textbf{8} \\ 
  pendigits-2 & 15 & 48 & 67 & \textbf{10} & 13 & 20 & 137 & --- & 12 \\ 
  satimage-2 & 15 & 15 & 18 & 8 & 9 & 10 & \scriptsize{525} & --- & \textbf{4} \\ 
  smtp & \scriptsize{3} & 21 & 34 & \scriptsize{6} & 7 & \scriptsize{19} & \scriptsize{17} & --- & \textbf{3} \\ 
  thyroid & \scriptsize{3} & 19 & 28 & 16 & 4 & 10 & 73 & --- & 8 \\ 
  tic-tac-toe & 25 & \scriptsize{412} & 411 & 29 & 42 & 81 & 102 & \scriptsize{20} & 29 \\ 
  vowels & 25 & \scriptsize{59} & 148 & 14 & \scriptsize{14} & 23 & 142 & --- & \textbf{9} \\ 
  waveform-2 & \scriptsize{4} & \scriptsize{166} & \scriptsize{406} & 19 & \scriptsize{21} & 79 & \scriptsize{522} & --- & 12 \\ 
  wdbc & \scriptsize{9} & 8 & \scriptsize{2} & \scriptsize{3} & \textbf{2} & \scriptsize{4} & \scriptsize{338} & --- & \textbf{2} \\ 
  anuran & --- & 81 & 1284 & 91 & \textbf{11} & 265 & 346 & --- & 100 \\ 
  avila & --- & 3460 & 7796 & 940 & 707 & \scriptsize{1297} & \scriptsize{181} & \scriptsize{8} & 726 \\ 
  car & --- & 660 & 776 & 56 & 172 & \scriptsize{72} & 308 & \scriptsize{14} & 132 \\ 
  contracept & --- & 1268 & \scriptsize{5536} & 12 & \scriptsize{15} & \scriptsize{222} & \scriptsize{5} & \scriptsize{14} & 13 \\ 
  drybeans & --- & 2482 & 7039 & \textbf{106} & 153 & \scriptsize{1281} & 202 & --- & 182 \\ 
  glass & --- & 25 & 20 & \textbf{5} & 14 & 8 & 171 & \scriptsize{12} & 6 \\ 
  heartcleveland & --- & \scriptsize{246} & \scriptsize{410} & 7 & \scriptsize{5} & 42 & \scriptsize{522} & \scriptsize{11} & 6 \\ 
  iris & --- & 15 & 13 & 3 & 6 & 8 & 27 & \scriptsize{9} & \textbf{2} \\ 
  pendigits & --- & 1275 & 1690 & 142 & 261 & 430 & 562 & \scriptsize{16} & 174 \\ 
  vehicle & --- & 561 & \scriptsize{760} & 25 & 47 & 176 & \scriptsize{741} & \scriptsize{11} & 23 \\ 
  waveform & --- & \scriptsize{2783} & \scriptsize{3520} & 126 & 143 & \scriptsize{970} & \scriptsize{76} & \scriptsize{6} & 161 \\ 
  wine & --- & 22 & 15 & 5 & 9 & 8 & 104 & \scriptsize{8} & \textbf{5} \\ 
   \hline
\end{tabular}
\end{table}
\begin{figure}[ht]
	\includegraphics[width=\textwidth]{heatmap_model_complexity}
	\caption{Heatmap for model complexity (higher is better): for each dataset, we divide the best (minimum) total number of literals by the total number of literals of each algorithm. We exclude the results from the models with much worse ROC-AUC scores (denoted as grey).}	 
	\label{fig:heatmap_modelcomplexity}
\end{figure}

\subsection{Model complexities}
The next question we study empirically is, does TURS produce more complex rule sets because it only allows overlaps formed by rules with similar probabilistic output? 

We first calculate the number of total literals for each model, by summing up the lengths of rules in a rule set, rule list, or decision tree (by treating each tree path as rule). 
%The total number of literals in a model represents the complexity if a domain expert would like to comprehend the model as a whole. 
We report this measure in Table~\ref{table:num_literal}. As simpler models with very much worse predictive performance is not so interesting, hence if a model's ROC-AUC score is more than $(-0.1)$ smaller than that of TURS, we exclude it from the following comparisons (denoted by \emph{smaller} font sizes in Table~\ref{table:num_literal}). 

We observe that TURS produces the simplest model for 13 out 31 datasets (excluding results from models with incomparable ROC-AUC scores), which we denote in bold in the table. 

Further, to illustrate the differences between the number of literals across all algorithms, we calculate a comparative score as follow: for each individual dataset, we divide the minimum total number of literals by the total number of literals of each algorithm, which we plot as a heat map in Figure~\ref{fig:heatmap_modelcomplexity}. From the heat map, CLASSY and TURS are obviously better than others in general; however, CLASSY induces ordered decision lists from data, with a much more complex internal logic than the unordered rule sets induced by TURS. 

%\textbf{Rule lengths.} For rule-based models with a large number of total literals, it is impractical for domain experts or analysts to comprehend the whole model by reading through all rules. In such cases, they may be more interesting to look at explanations for single predictions; thus, shorter rules are preferred. 
%\todo[inline]{What is the message for rule lengths? Maybe do not report this?}

\subsection{Ablation study: local constraint}
\begin{table}[ht]
\small
\centering
\begin{tabular}{llllll}
  \hline
Constraint & \# rules & rule length & ROC-AUC & MDL score & train/test prob. diff. \\ 
  \hline
No & 12.48($\pm$1.56) & 5.597($\pm$0.42) & 0.722($\pm$0.02) & 2191.189($\pm$65.91) & 0.049($\pm$0.01) \\ 
  Yes & 1($\pm$0) & 1($\pm$0) & 0.724($\pm$0.01) & 2050.087($\pm$68.88) & 0.007($\pm$0) \\ 
   \hline\end{tabular}
\caption{Results of ablation study on whether to use the local constraint. We report the mean ($\pm$ standard deviation) over $100$ repetitions. } \label{table:ablation_simu}
\end{table}
\begin{figure}[ht]
	\includegraphics[width=\textwidth]{ablation_local_constraint}
	\caption{The process of rules being added to the rule set, with and without the local constraint heuristics, using the first  dataset among the 100 simulated datasets. Each point represent the status when after a single rule is added, with x-axis representing the coverage of the (potentially incomplete) rule set after adding this rule, and y-axis representing the MDL-based score. } \label{fig:ablation_simulation}
\end{figure}
We consider a simple simulation study to illustrate the necessity of the local constraint, where our feature variables are denoted as $X = (X_1, ..., X_{50})$. Assume all variables in $X$ are binary, and we sample $X_1 \sim Ber(0.2)$, $X_i \sim Ber(0.5) (i = 2, ..., 50)$, in which $Ber(.)$ denotes the Bernoulli distribution. Further, we consider binary target variable $Y$ and sample $Y|X_1 = 1 \sim Ber(0.7)$ and $Y|X_1 = 0 \sim Ber(0.95)$. That is, $X_1 = 1$ (or $X_1 = 0$) is the only ``rule" in this simulated dataset. 

We simulate the dataset with the sample size $5,000$ for $100$ times, and run TURS with and without the local constraints. As shown in Table~\ref{table:ablation_simu}, without the local constraint, we will achieve worse (bigger) MDL-based score. 

Notably, when not using the local constraint, the number of rules  and the rule lengths are both not consistent with the ``true" model, showing that irrelevant variables are picked when growing the rules. We have two perspectives to explain the inconsistency. 

To begin with, when the local constraint is \emph{not} imposed, the difference between the class probabilities estimated from the training and test dataset is larger than the difference when the local constraint is imposed, showing that the rules as local probabilistic models generalize worse when the local constraint heuristic is turned off. That is, we observe overfitting locally. Further, as we write as motivation in Section~\ref{subsec:local_constraint}, the local constraint heuristic is designed to prevent leaving out instances that are difficult to cover for `future' rules, we do notice this phenomenon empirically. Specifically, for a single run of TURS on the simulated dataset, we plot the process of rules being added to the (potentially incomplete) rule set, with x-axis showing the coverage of the rule set, versus the MDL-based model selection criterion for the rule set. 

We plot in Figure~\ref{fig:ablation_simulation} the procedure of iteratively searching for the next best rule: each point represents the status of the rule set after a single rule is added, with x-axis representing the coverage of the rule set (i.e., the number of instances covered by at least one of the rules), and y-axis representing the MDL-based score for the rule set as a whole model. Thus, our compression learning rate heuristic, defined in Section~\ref{subsec:learning_rate}, basically tries to iteratively find the next point (i.e., next rule) in Figure~\ref{fig:ablation_simulation} with the steepest slope. However, without the local constraint heuristic, the search for the next rule conducted by the algorithm may become too greedy: when growing a rule and consequently reducing the rule's coverage, the instances left out to be covered by future rules are simply ignored, which leads to inferior optimization results in the end (as shown by the red curve in Figure~\ref{fig:ablation_simulation}). 

\subsection{Ablation study: diverse coverage beam search}
\begin{figure}[ht] \label{fig:diverse_coverage}
	\includegraphics[width=\textwidth]{ablation_diverse_coverage}
	\caption{Ablation study: the ROC AUC on the test set, averaged over the five-fold stratified cross-validation.}
	\label{fig:heatmap_modelcomplexity}
\end{figure}
We study the effect of using the beam search with the ``diverse coverage", by replacing it with the ``normal" beam search. Suppose the beam width is $W$, we simply pick the best $W$ rule growth candidates \emph{without categorizing rule growth candidates by their coverage}. That is, we ``turn off" the diverse coverage constraints both for the updating the beam and the auxiliary beam. 

As shown in Figure~\ref{fig:diverse_coverage}, when using the diverse coverage heuristic, the ROC-AUC on the test sets (points and curve in orange) becomes better on 25 out of 31 datasets, demonstrating the benefits for the predictive performance. 

\subsection{Runtime}
\begin{figure}[ht] \label{fig:runtime}
	\includegraphics[width=\textwidth]{runtime}
	\caption{Runtime comparison among rule set methods.}	 
	\label{fig:heatmap_modelcomplexity}
\end{figure}
Last, we report the runtime of TURS, together with all rule set competitor methods only, as decision trees/lists methods from mature software (Weka and Python Scikit-Learn) are highly optimized in speed and are known to be very fast. 

We illustrate the runtime (in seconds), averaged over the five-fold cross-validations, in Figure~\ref{fig:runtime} (note that the y-axis is in scaled by $log_{10}$). In general, the runtime of TURS are competitive among all rule set methods, with only CN2 showing clear superiority. CN2 seems faster in general and scales better to larger datasets, which can be caused both by a more efficient implementation (from the software ``Orange"), and by its algorithmic properties (a greedy and separate-and-conquer approach). 

%In addition, for 22 out of 31 datasets, the runtime of TURS (in purple) are less than 100 seconds. On the other hand, TURS needs more than $10,000$ seconds on 6 of them (backdoor, magic, avila, drybeans, pendigits, waveform). 








%\subsection{Overlaps' insignificance}
%\begin{figure}
%	\includegraphics[width=\textwidth]{overlap_sig}
%\end{figure}






% Acknowledgements and Disclosure of Funding should go at the end, before appendices and references
\section{Conclusion}
We studied the problem of learning unordered probabilistic rule sets without implicit orders, with the intuitive idea by only allowing rules to overlap only if they have similar probabilistic outputs. We further formalized rule sets as probabilistic models in a principled way, in which the core trick is to estimate the class probabilities for instances covered by multiple rules by ``taking the union" of all these rules. Next, we adopted a model selection approach by designing an MDL-based model selection criterion, together with a carefully designed heuristic algorithm, which have been shown to induce rules with competitive performance with respect to prediction, model complexity, (probabilistic) generalization to unseen instances, and uniquely, being truly unordered. 

For future work, we will study the practical use of our method with a case study in the health care domain. This involves a user study to investigate whether, and in what degree, the domain experts find the truly unordered property of rule sets obtained by our method helps them comprehend the rules better in practice, in comparison to rule lists/sets with explicit or implicit orders.

\acks{This work is part of the research program ‘Human-Guided Data Science by Interactive Model Selection’ with project number 612.001.804, which is (partly) financed by the Dutch Research Council (NWO).}

% Manual newpage inserted to improve layout of sample file - not
% needed in general before appendices/bibliography.

\newpage
%
\appendix
\section*{Proof of Proposition 1}
%\begin{proposition}
%Given a rule set $\ruleset$ in which for any $S_i, S_j \in \ruleset$, $S_i \cap S_j = \emptyset$, then $P^{NML}_{\ruleset}(Y^n=y^n|X^n=x^n) = P^{apprNML}_{\ruleset}(Y^n=y^n|X^n=x^n)$.
%\end{proposition}
\primelemma*

\begin{proof}
The numerators are the same, and hence we only need to show that the denominators are the same. Assume there are $K$ rules in $M$ in total, 
\begin{equation} 
\begin{split}
		& \sum_{z^n \in \mathscr{Y}^n} P_{M, \htheta(x^n, z^n)}(z^n|x^n) = \sum_{z^n} \prod_{S\in M} \hat{P}_S(y^S|X^S) \\
		& = \sum_{z^n} \hat{P}_{_{S_1}}(z^{S_1}|x^{S_1}) \ldots \hat{P}_{_{S_K}}(z^{S_{K}}|x^{S_{K}}) \\
		& = \sum_{z^{S_1}} \ldots \sum_{z^{S_{K}}} \left(\hat{P}_{_{S_1}}(z^{S_1}|x^{S_1}) \ldots \hat{P}_{_{S_K}}(z^{S_{K}}|x^{S_{K}}) \right)\\
		& = \Bigg(\sum_{z^{S_1}} \ldots \sum_{z^{S_{K-1}}} \hat{P}_{_{S_1}}(z^{S_1}|x^{S_1}) \ldots \hat{P}_{_{S_{K-1}}} (z^{S_{K-1}}|x^{S_{K-1}}) \Bigg) \left(\sum_{z^{S_{K}}}  \hat{P}_{_{S_K}}(z^{S_{K}}|x^{S_{K}})\right) \\
				& \ldots \\
		& = \left(\sum_{z^{S_{1}}}  \hat{P}_{_{S_{1}}}(z^{S_{1}}|x^{S_{1}})\right) \ldots \left(\sum_{z^{S_{K}}}  \hat{P}_{_{S_K}}(z^{S_{K}}|x^{S_{K}})\right)\\
		& = \prod_{S \in M} \sum_{z^{S}}  \hat{P}_{_{S}}(z^{S}|x^{S}) \\
		& = \prod_{S \in M} \mathcal{R}(|S|, |\mathscr{Y}|),
\end{split}
\end{equation}
which completes the proof.
\end{proof}

\section*{Proof of Proposition 2}
%\begin{proposition}
%Assume $\ruleset$ contains $K$ rules in total, including the else rule, and we have $n$ instances. Then $
%\log \left(\prod_{S \in \ruleset} \mathcal{R}(|S|, |\mathscr{Y}|)\right) = \frac{K(|\mathscr{Y}| - 1)}{2} \log n + \mathcal{O}(1)$, where $\mathcal{O}(1)$ is bounded by a constant w.r.t.\ to $n$.
%\end{proposition}
\secondlemma*

\begin{proof}
	The proof directly follows from Theorem~3 of~\citep{silander2008factorized}. Firstly, it has been proven that $\log \mathcal{R}(|S|, |\mathscr{Y}|) = \frac{|\mathscr{Y}| - 1}{2} \log |S| + \mathcal{O}(1)$~\citep{rissanen1996fisher}. Next, under the mild assumption that $|S|$ grows linearly as the full sample size $n$, we have $\log |S| = \log ((\gamma + o(1))n) = \log n + \mathcal{O}(1)$. Hence, $\log \prod_{S \in M} \mathcal{R}(|S|, |\mathscr{Y}|) = \sum_{S}\log \mathcal{R}(|S|, |\mathscr{Y}|)$ $= \frac{K(|\mathscr{Y}| - 1)}{2} \log n + \mathcal{O}(1)$, which completes the proof. 
\end{proof}

%\section{}
%\label{app:theorem}
%
%% Note: in this sample, the section number is hard-coded in. Following
%% proper LaTeX conventions, it should properly be coded as a reference:
%
%%In this appendix we prove the following theorem from
%%Section~\ref{sec:textree-generalization}:
%
%In this appendix we prove the following theorem from
%Section~6.2:
%
%\noindent
%{\bf Theorem} {\it Let $u,v,w$ be discrete variables such that $v, w$ do
%not co-occur with $u$ (i.e., $u\neq0\;\Rightarrow \;v=w=0$ in a given
%dataset $\dataset$). Let $N_{v0},N_{w0}$ be the number of data points for
%which $v=0, w=0$ respectively, and let $I_{uv},I_{uw}$ be the
%respective empirical mutual information values based on the sample
%$\dataset$. Then
%\[
%	N_{v0} \;>\; N_{w0}\;\;\Rightarrow\;\;I_{uv} \;\leq\;I_{uw}
%\]
%with equality only if $u$ is identically 0.} \hfill\BlackBox
%
%\section{}
%
%\noindent
%{\bf Proof}. We use the notation:
%\[
%P_v(i) \;=\;\frac{N_v^i}{N},\;\;\;i \neq 0;\;\;\;
%P_{v0}\;\equiv\;P_v(0)\; = \;1 - \sum_{i\neq 0}P_v(i).
%\]
%These values represent the (empirical) probabilities of $v$
%taking value $i\neq 0$ and 0 respectively.  Entropies will be denoted
%by $H$. We aim to show that $\fracpartial{I_{uv}}{P_{v0}} < 0$....\\
%
%{\noindent \em Remainder omitted in this sample. See http://www.jmlr.org/papers/ for full paper.}


\vskip 0.2in
\bibliography{main}

\end{document}
